\section{Implementation and Evaluation}
\label{sec:impl}

The implementation of the algorithm has been added to a branch of the  \jkind~\cite{jkind} model checker~\footnote{The \jkind fork with \jsynvg is available at \url{https://goo.gl/WxupTe}}.  \jkind officially supports synthesis
%\grigory{Unofficially? We have the Tech Report, it is all official ;)}
using a $k$-inductive approach, named \jsyn~\cite{KatisFGBGW16}. For clarity, we named
our validity-guided technique \jsynvg (i.e., validity-guided synthesis). \jkind uses Lustre~\cite{lustrev6} as its specification and implementation language.
\iffalse
, which functions as an intermediate representation to the Architecture Analysis and Design Language (\textsc{AADL})~\cite{feiler2006architecture}.
The latter is a high-level specification and analysis language with which
contracts are expressed, using the Assume-Guarantee Reasoning (\textsc{AGREE})
framework~\cite{NFM2012:CoGaMiWhLaLu}.
\andrew{it seems strange to be talking about AADL and maybe even AGREE  at all here.}

\fi
%
%\footnote{An unofficial release of \jkind
%including our synthesis algorithm is available to download at
%https://github.com/andrewkatis/jkind-1/tree/synthesis. \aeval needs to be
%installed separately from https://github.com/grigoryfedyukovich/aeval.}.
%During the internal process,
\jsynvg encodes Lustre specifications in the language of
linear real and integer arithmetic (LIRA)
% first order logic
and communicates them to \aeval~\footnote{\aeval is available at~\url{https://goo.gl/CbNMVN}}.
%f the SMT-LIB language,
%, with which the $\forall\exists$-formulas, regions of
%validity, as well as the witnesses are expressed.
%The underlying solver that is
%used is the \aeval Skolemizer,
%which currently supports
%$\forall\exists$-formulas expressed using
%linear real and integer arithmetic (LIRA).
%
%
%For realizable specifications, and considering Alg.~\ref{alg:synthesis}, we
%recursively use regions of validity to block out unsafe states from the search
%space.
%Eventually, we reach a fixpoint that is sufficient to use in order to extract a
Skolem functions returned by \aeval get then translated %that witnesses the realizability of the specification. Given
%the generated function, what remains is its translation
into an efficient and practical implementation. To compare the quality of implementations against \jsyn, we use
\smtlibtoc, a tool that has been specifically developed to translate
  Skolem functions to C implementations~\footnote{The \smtlibtoc tool is available at \url{https://goo.gl/EvNrAU}}.
%\grigory{There used to be a rehash of Alg.~\ref{alg:synthesis}; no need to repeat it here.}
%\grigory{Also, you probably need to refer the reader that more info about implementation is in~\cite{KatisFGBGW16}.}

%In future work,
%we intend to further improve on this, and
%many other aspects of the compiler's performance as part of an individual
%future work.



%%% Local Variables:
%%% mode: latex
%%% TeX-master: "document"
%%% End: 