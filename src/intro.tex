
\section{Introduction}
\andreas{The following paragraph is the same with the one in the ATVA paper}
Automated synthesis research is concerned with discovering efficient algorithms to construct programs that are guaranteed to comply with predefined temporal specifications.
This problem has been well studied for propositional specifications, especially for (subsets of) LTL~\cite{gulwani2010dimensions}.
More recently, the problem of synthesizing programs for richer theories has been
examined, including work in {\em template-based
synthesis}~\cite{srivastava2013template}, which attempts to find programs that
match a certain shape (the template), and {\em inductive synthesis}, which
attempts to use counterexample-based refinement to solve synthesis problems~\cite{flener2001inductive}.  Such techniques have been widely used for stateless formulas over arithmetic domains~\cite{reynolds2015counterexample}.
\textit{Functional synthesis} has also been effectively used to synthesize
subcomponents of already existing partial
implementations~\cite{kuncak2013functional}.

In this work, we describe a novel approach that can effectively synthesize
programs using specifications written in the form of an arbitrary {\em
assume/guarantee contract}. The technique has been excersised to specification
that includes safety properties in linear real and integer arithmetic (LIRA),
but remains generic enough to be extended into supporting additional theories
in the future. This is a ``hands-off'' approach, as there is no
requirement from the user to interact with the underlying machinery,
\andreas{I edited the way multiple citations look (all in one pair of
brackets). I'm not sure if this is allowed, so please change it if you know.}
unlike~\cite{ryzhyk2014user,ryzhyk2016developing}, and is capable of providing solutions without the guidance of templates, like
in~\cite{beyene2014constraint}.

The main idea of the algorithm was inspired by the IC3 algorithm, its
technique otherwise known as Property Directed Reachability
(PDR)~\cite{bradley2011sat,een2011efficient}. In PDR, one aims to discover an
inductive invariant for a property, by recursively blocking generalized regions describing unsafe states. In a similar concept, we attempt
to reach a greatest fixpoint that contains states usable indefinitely by the
system, in order to react to unpredictable environment behaviour, while
complying to well-defined specification. As such, beginning from the entire
problem space, we recursively block regions of states that violate the contract, using \textit{regions of validity} that are
generated by non-valid $\forall\exists$ formulas. If the refined
$\forall\exists$ formula is valid, we reach an approximation of a fixpoint which can effectively be used by the specified transition relation, to
provide safe reactions to environment inputs. We then extract a witness for the
formula's satisfiability, which can be directly transformed into the
language intended for the system's implementation.

The algorithm was implemented as a feature to the \jkind model checker, which
already had unofficial support to program synthesis according to work based on
k-inductive proofs of a contract's
realizability~\cite{gacek2015towards,katis2016towards,katis2016synthesis}.
Our approach is completely independent of the k-inductive method, but is still
based on the extraction of a witness that satisfies a $\forall\exists$ formula, using the \aeval Skolemizer~\cite{fedyukovich2015automated}. To extract such a
witness, we do not depend on a k-inductive proof, but instead use the
\textit{regions of validity} that \aeval can generate from non-valid formulas,
to reach a fixpoint of satisfiable assignments to state variables.
This approach is a direct improvement over the k-inductive method in two
important aspects; performance, and soundness of 'unrealizable' results. While
the former is self-explanatory, with the latter we refer to cases where the
k-inductive algorithm would spuriously report a contract as 'unrealizable', when a correct
implementation actually exists. This unsoundness stems from the pessimistic
behavior of the k-inductive algorithm, as it is not capable to only consider a
``safe'' subset of the state space. We were able to confirm our claims
by comparing the two algorithms under a comprehensive benchmark suite containing
contracts that were initially used in verification problems, as well as
specification for industrial-level designs.

In Section~\ref{sec:synthesis} we provide the background that is required in the
context of this paper, and present the validity-guided approach to synthesizing
implementations. Section~\ref{sec:aeval} contains information regarding \aeval's
capability to compute valid subsets from non-valid $\forall\exists$ formulas.
An outline of the algorithm's implementation is described
in Section~\ref{sec:impl}, and we show the advancements of our validity-guided
approach against the k-inductive method in Section~\ref{sec:results}.
We discuss the differences of our work with closely related approaches In
Section~\ref{sec:related}. Finally, we discuss potential future work and
conclude in Section~\ref{sec:conclusion}.

