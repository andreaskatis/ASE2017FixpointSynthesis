\section{Validity-Guided Synthesis from Assume-Guarantee Contracts}
\label{sec:synthesis}


In this section we define Assume-Guarantee contracts (Sect.~\ref{sec:pre}),
describe the validity-guided approach we take towards synthesizing
implementations (Sect.~\ref{sec:synth}),
and finally illustrate how it works using the popular cinderella-stepmother
game (Sect.~\ref{sec:example}).
%Finally, we enrich our formal definitions with an informal proof of the
%algorithm's correctness in terms of the successfully synthesized
%implementations.

\subsection{Assume-Guarantee Contracts}
\label{sec:pre}

For the purposes of this paper, we focus our interest in a mainstream variation
for representing system requirements, using an \textit{Assume-Guarantee
Contract}. Requirements in this format contain two main types of constraints.
The \emph{assumptions} of the contract restrict the possible inputs that the
environment can provide to the system, while the \emph{guarantees} are used to
describe what is considered a safe reaction of the system to the outside world. 

A simple example is the contract with assumption $A = \{x\neq
y\}$ and guarantee $G = \{x \leq y \Rightarrow z =
\textit{true}, x \geq y \Rightarrow z = \textit{false}\}$. Variables
$x$ and $y$ are the designated inputs while $z$ is the output. This is a
well-defined contract, since by the assumption $A$, it is always the case that
$x \neq y$, therefore at least one implementation exists, which, for example
sets $z$ to true if $x < y$ and false otherwise. An alternative
valid implementation could set $z$ to false if $x > y$, and true otherwise. The
proof of existence of such an implementation is the main concept behind the
\emph{realizability} problem, while the automated construction of a witness
implementation is the main focus of \emph{program synthesis}.


It is apparent that the example contract above is therefore \emph{realizable},
and an efficient synthesis procedure would be capable of providing at least one
implementation. Nevertheless, it is important to consider a variation of the
example, where $A = \emptyset$. This is a practical case of an
\emph{unrealizable} contract, as there is no feasible implementation that can
correctly react to the environment assigning values to $x$ and $y$, such that
$x = y$.

\subsection{Formal Representation}
\label{sec:formals}
We use two disjoint sets, $state$ and $inputs$, to describe a system.
A straightforward and intuitive way to represent an \emph{implementation} is by
defining a \emph{transition system}, composed of an initial state
predicate $I(s)$ of type $state \to bool$, as well as a transition relation
$T(s,i,s')$ of type $state \to inputs \to state \to bool$.

Combining the above, we represent an Assume-Guarantee (AG) contract using a set
of \emph{assumptions}, $A: state \rightarrow inputs \rightarrow bool$,
and a set of \emph{guarantees} $G$. The latter is further decomposed into two
distinct subsets $G_I: state \rightarrow bool$ and $G_T: state \rightarrow
inputs \rightarrow state \rightarrow bool$. $G_I$ defines the set of valid
initial states, and $G_T$ contains constraints that need to be satisfied in
every transition between two states. An important note at this point is that we
we do not make any distinction between internal state variables and outputs in the
formalism. This alone allows us to use state variables to (in some cases)
simplify specification of guarantees, since we do not expect a contract
to be always defined over all variables in the transition system.

Consequently, we can formally define a realizable contract, as one for which any
preceeding state $s$ can make a transition into a new state $s'$ that satisfies
the guarantees, assuming valid inputs. For a system to be ever-reactive, these
new states $s'$ should be further usable as preceeding states in a future
transition. States like $s$ and $s'$ are defined as being \textit{viable}, if
and only if:

\begin{equation}
\forall s,i.~ A(s, i) \land \viable(s) \Rightarrow \exists s'.~ G_T(s, i,s')
\land \viable(s')
\label{eq:viable}
\end{equation}

A necessary condition, finally, is that the set of viable states has to
intersect with the set of initial states. As such, to conclude that a contract
is realizable, we require that

\begin{equation}
\viable(s) \land G_I(s) \neq \emptyset
\label{eq:nonempty}
\end{equation}

Therefore, the intuition behind our proposed algorithm relies on the discovery
of a greatest fixpoint that only contains viable states.


\subsection{Validity-Guided Synthesis from Contracts}
\label{sec:synth}

The main contribution presented in this paper, is a novel idea that effectively
uses the information provided by \textit{regions of validity} to compute a
greatest fixpoint of safe states. In our context, this fixpoint is not only
usable as a proof to the realizability of a specification, but also leads to the
construction of a witness that can be translated with a straightforward process
into a functional and efficient implementation.

The algorithm~\ref{alg:synthesis} shows this validity-guided technique.
We initialize the process by defining the fixpoint $F(s)$ to be equal to $true$.
Eventually, the algorithm attempts to converge to such a fixpoint $F(s)$ that
only contains viable states, considering Equation~\ref{eq:viable}. We therefore
construct the formula $\phi$, and provide it as an input to \aeval, an efficient
Skolemizer for $\forall\exists$ formulas. \aeval is particularly focused on
determining the validity of $\phi$. If the formula is valid, then a witness
\textit{Skolem} is constructed, containing valid assignments to the
existentially quantified variables of $\phi$. In the context of viability, this
witness is capable of providing viable states that can be used as a safe
reaction, considering the precedence of a viable state and an input that
satisfies the assumptions.

If $\phi$ is not true for every possible assignment of the universally
quantified variables, \aeval provides an exact subset of $F(s) \land A(s,i)$, namely
$Q(s,i)$, which, if plugged in the original left-hand side of $\phi$, makes the
resulting formula valid. We call $Q(s,i)$ a \textit{region of validity}.

At this point, one could falsely assume that replacing $F(s) \land A(s,i)$ with
$Q(s,i)$ is sufficient to solve our problem, and use the resulting witness as a
candidate implementation. This is not the case however, as $Q(s,i)$ is a subset
of both state and input variables. As such, it may contain further constraints
over the contract's inputs. This would lead to implementations that only
consider a subset of the original assumptions of the contract, with no
pre-defined strategy for the rest of the originally valid inputs.
Fortunately, we can exploit \aeval's capability of providing regions of validity
towards eliminating this issue.

The main concept to properly refine $F(s)$, is to extract a region of validity
that only involves constraints over state variables. To achieve this, we ask for
the validity of the formula $\phi' \gets \forall s. F(s) \Rightarrow \exists
i. A(s,i) \land \lnot Q(s,i)$. If $\phi'$ is a valid formula, then for any
assignment of the state variables $s$, we have a valid input (i.e. that
satisfies the assumptions), for which the states are taken outside of the region
of validity $Q(s,i)$. This is a case of an unrealizable contract, as no state is
safe in this context. On the other hand, if $phi'$ is not valid, \aeval computes
a new region of validity, namely $W(s)$. The new region is a strict subset of
$F(s)$, is described using constraints over state variables only, and entails
the existence of unsafe states in $F(s)$, considering valid inputs.

Having this new region of validity, we can finally refine $F(s)$, by conjucting
to it the negation of $W(s)$. Despite the fact that we now have a refined
candidate fixpoint, we are not yet done, as $\lnot W(s)$ is not an exact region
of validity, with respect to $Q(s,i)$. As such there might still be states in
$Q(s,i)$ that are not covered by $\lnot W(s)$. Therefore, we reiterate the
process by repeating the top-level \aeval query, with $F(s) = F(s) \land \lnot
W(s)$. Eventually, we either reach a greatest fixpoint $F(s)$ that effectively
describes the set of viable states, or reach the case where $F(s) = false$, and
declare our contract to be unrealizable.

As a final remark, \aeval's effectiveness in providing witnesses to the
satisfiability of $\forall\exists$ formulas is also exploited in terms of the
tool providing concrete counterexamples to unrealizable contracts, using line 19
in Algorithm~\ref{alg:synthesis}. In this particular case, if $\phi'$ is a valid
formula, we can extract a witness that can be essentially used as a test case to
prove the current specification unrealizable. The witness contains certain
assignments to input variables, for which the condition of viability does not
hold, for any state. We leave the specifics regarding the meaning and usability
of such counterexamples as potential future work.


\begin{algorithm2e}[t!]
\SetAlgoSkip{}
\SetKwFor{For}{for}{do}{}
\KwOut{$Result: \{\realizable, \unrealizable\}$, 
%\textcolor{red}{\init: state},
\skolems: Skolem Function for implementation or cex
}
\BlankLine
$\skolems \gets null$; \\
%$InitResult \gets $\sc{Sat?}$(G_I)$; \\
% \uIf(\label{alg:initState}){$(\isUnsat(InitResult))$}
% 	{%
% 		\Return
% 		\unrealizable, $\emptyset$, $\langle \rangle$;%
% 	}
%\textcolor{red}{$\init \gets InitResult.model$;} \\
$F(s) \gets true$;\\	
\While{true}{
$\phi \gets \forall s,i. F(s) \land A(s,i) \Rightarrow \exists s'. G_{T}(s,i,s')
\land F(s')$;
\\
$aevalResult \gets \aeval(\phi)$;\\
\uIf(\label{alg:returnSat}){$(\isValid(aevalResult))$}
{
	$\skolems \gets (aevalResult.\skolems)$;\\
	\uIf(){$G_{I}(s) \land F(s) \neq false$}
	{
	\Return \realizable, %\textcolor{red}{\init},
	 \skolems;
	}\uElse(){
	 \Return \unrealizable;
	}
}
\uElse
	{%
		$Q(s,i) \gets aevalResult.validSubset$;\\
		\uIf{$Q(s,i) = false$}{
			\Return \unrealizable;
		}
		$\phi' \gets \forall s. F(s) \Rightarrow \exists i. A(s,i) \land \lnot
		Q(s,i)$;\\
		$aevalResult' \gets \aeval(\phi')$;\\		
		\uIf(\label{alg:returnSat}){$(\isValid(aevalResult'))$}
		{
			$\skolems \gets (aevalResult'.\skolems)$;\\			
			\Return \unrealizable, \skolems;
		}\uElse{
			$W(s) \gets aevalResult'.validSubset$;\\
			$F(s) = F(s) \land \lnot W(s)$;
		}
}
}
\caption{Validity-Guided Synthesis}
\label{alg:synthesis}
\end{algorithm2e}

\subsection{An Illustrative Example}
\label{sec:example}