\section{Validity-Guided Synthesis from Assume-Guarantee Contracts}
\label{sec:synthesis}

\begin{algorithm*}[!t]
\caption{\jsynvg (A : assumptions, G : guarantees)}
\label{alg:synthesis}
\begin{algorithmic}[1]
%\Procedure{\jsynvg}{A : assumptions, G : guarantees}
	\State $F(s) \gets true$;\Comment{Fixpoint of viable states}
	\While{true}
		\State $\phi \gets \forall s,i. \ (F(s) \land A(s,i) \Rightarrow \exists s'.G_{T}(s,i,s') \land F(s'))$;
		\State $\tuple{\skolems, \subs} \gets \aeval(\phi)$;
		\If{$\isValid(\phi)$}
			\If{$G_{I}(s) \land F(s) \neq false$}
				\Return $\tuple{\realizable, \skolems}$;
			\Else{\Comment{Empty set of initial states}}
		 		\Return $\tuple{\unrealizable, \varnothing}$;
		 	\EndIf
		\Else{\Comment{Extract region of validity $Q(s,i)$}}
			\State $Q(s,i) \gets \subs$;
			\If{$Q(s,i) = false$}
				\Return $\tuple{\unrealizable, \varnothing}$;
			\EndIf
			\State $\phi' \gets \forall s. \ (F(s) \Rightarrow \exists i. A(s,i) \land \lnot
			Q(s,i))$;
			\State $\tuple{\skolems', \subs'}\gets \aeval(\phi')$;
			\If{$(\isValid(\phi'))$}\Comment{If $Q(s,i)$ contains only unsafe states}
				\Return $\tuple{\unrealizable, \skolems'}$;
			\Else{}\Comment{Extract witnessing cex}
				\State $W(s) \gets \subs'$;
				\If{$W(s) = false$}
					\Return $\tuple{\unrealizable, \varnothing}$;
				\EndIf
				\State $F(s) \gets F(s) \land \lnot W(s)$;\Comment{Refine set of viable states}
			\EndIf
		\EndIf
	\EndWhile
%\EndProcedure
\end{algorithmic}
\end{algorithm*}

The main contribution presented in this paper, is a novel idea that effectively
uses the information provided by regions of validity to compute a
greatest fixpoint of safe states. In our context, this fixpoint is not only
usable as a proof to the realizability of a specification, but also leads to the
construction of a witness that can be translated with a straightforward process
into a functional and efficient implementation. The algorithm is
presented in detail in Section~\ref{sec:alg} and we provide soundness
proofs regarding the algorithm's capabilities of determining the
realizability (unrealizability) of specification in
Section~\ref{sec:soundness}. An illustrative example is used in
Section~\ref{sec:example} to guide the reader through the process of
synthesizing implementations, using a specification that describes the Cinderella-Stepmother game.

\subsection{Algorithm}
\label{sec:alg}
Algorithm~\ref{alg:synthesis}, named \jsynvg, shows the validity-guided technique that we use towards the automatic synthesis of implementations. The specification is written using the Assume-Guarantee convention that we described in Section~\ref{sec:formals} and is provided as an input to the algorithm. 
Line 1 initializes the process by defining the fixpoint $F(s)$ to be equal to
$true$. The algorithm then attempts to converge to a fixpoint $F(s)$
that only contains viable states, considering Equation~\ref{eq:viable}
and~\ref{eq:nonempty}.
We therefore construct the formula $\phi = \forall s,i. \ (F(s) \land A(s,i)
\Rightarrow \exists s'. G_{T}(s,i,s') \land F(s'))$, and provide it as an input to \aeval (Line 4). \aeval is particularly
focused on determining the validity of $\phi$. If the formula is valid, then a witness
\textit{Skolem} is constructed, containing valid assignments to the
existentially quantified variables of $\phi$. In the context of viability, this
witness is capable of providing viable states that can be used as a safe
reaction, given a prior viable state and an input that
satisfies the assumptions.

If $\phi$ is not true for every possible assignment of the universally
quantified variables, \aeval provides a \textit{region of validity} $Q(s,i)$
(Line 11).
%an exact subset of $F(s) \land A(s,i)$, namely $Q(s,i)$, which, if plugged in
% the original left-hand side of $\phi$, makes the resulting formula valid. We will refer to such subsets as \textit{regions of
%validity}.
At this point, one could falsely assume that replacing $F(s) \land A(s,i)$ with
$Q(s,i)$ is sufficient to solve our problem, and use the resulting witness as a
candidate implementation. This is not the case however, as $Q(s,i)$ is a subset
of both state and input variables. As such, it may contain further constraints
over the contract's inputs. This would lead to implementations that only
consider a subset of the original assumptions of the contract, with no
pre-defined strategy for the rest of the originally valid inputs.
Fortunately, we can exploit \aeval's capability of providing regions of validity
towards eliminating this issue.

The main concept to properly refine $F(s)$, is to extract a region of validity
that only involves constraints over state variables. To achieve this, we ask for
the validity of the formula $\phi' = \forall s. \ (F(s) \Rightarrow \exists
i. A(s,i) \land \lnot Q(s,i))$ (Line 17). If $\phi'$ is a valid formula, then for
any assignment of the state variables $s$, we have a valid input (i.e. that
satisfies the assumptions), for which the states are do not satisfy the constraints defined by the region
of validity $Q(s,i)$. This is a case of an unrealizable contract, as no state is
safe in this context. On the other hand, if $\phi'$ is not valid, \aeval computes
a new region of validity, namely $W(s)$. The new region is a strict subset of
$F(s)$, is described using constraints only over state variables, and entails
the existence of unsafe states in $F(s)$, considering valid inputs.

Having a new region of validity that is not equal to \textit{false} (Line 23),
we can finally refine $F(s)$, by conjucting to it the negation of $W(s)$ (Line 25). Despite the fact that we now have a refined set of viable states, we are not yet done,
as $\lnot W(s)$ is not an exact region of validity, with respect to $Q(s,i)$. As
such there might still be states in $Q(s,i)$ that are not covered by $\lnot W(s)$. Therefore, we reiterate the
process by repeating the top-level \aeval query, with $F(s) = F(s) \land \lnot
W(s)$. \john{I'm not sure what notation you want to use here, but $=$ does not make sense since it is already being used for logical equality. Maybe use $\leftarrow$ or $:=$?} Eventually, we either reach a greatest fixpoint $F(s)$ that effectively
describes the set of viable states, or reach the case where $F(s) = false$, and
declare our contract to be unrealizable.

\aeval's effectiveness in providing witnesses to the
satisfiability of $\forall\exists$-formulas is also exploited in terms of the
tool providing concrete counterexamples to unrealizable contracts, using line 19
in Algorithm~\ref{alg:synthesis}. In this particular case, if $\phi'$ is a valid
formula, we can extract a witness that can be essentially used as a test case to
demonstrate the specification's unrealizability. The witness contains
certain assignments to input variables, for which the condition of viability does not
hold, for any state. We leave the details regarding the meaning and usability
of such counterexamples as potential future work.

\subsection{Soundness}
\label{sec:soundness}
To prove Algorithm~\ref{alg:synthesis}'s soundness regarding results, we first need to show that the algorithm always computes a fixpoint containing only state variable assignments that lead to the satisfiability of $\forall\exists$-formulas following the form of Equation~\ref{eq:viable}. To achieve this, we
use the fixed point theorem that was first stated by Alfred Tarski~\cite{tarski1955lattice}.


\andreas{An alternative would be to define a set S where each element is a set X of valid regions that lead to SAT, and the order in which X elements are constructed to be the relation that establishes a partial order over S. I have included these alternative theorems in an iffalse block in the .tex file. Please let me know which seems better tou you.}

\iffalse
We define the set $S = \{M_k, \ k \in \mathbb{N} \land \forall s, i. \ (M_k(s) \land A(s,i) \Rightarrow \exists s'. G_T(s,i,s') \land M_k(s'))\}$. In other words, $S$ is the set of subsets $M_k$, $k = 0, 1 , \ldots$ that contain constraints to states $s$, such that the formula $\forall s, i. \ (M_k(s) \land A(s,i) \Rightarrow \exists s'. G_T(s,i,s') \land M(s'))$ is satisfiable. The binary relation $\subseteq$ establishes a partial order on $S$, such that for any $k$, $M_k \subseteq M_{k+1}$.

\begin{lemma} The system $\mathfrak{U} = \tuple{S, \subseteq}$ is a complete lattice, where every subset $S' \subseteq S$ has a greatest lower bound  $\glb(S') = \cap S'$  and a least upper bound $\lub(S') = \cup S'$.
\label{lem:altlattice}
\end{lemma}
\begin{proof}
Since the inclusion relation $\subseteq$ has been established over $S$, every subset $S' = \{M_k, M_{k+1}, \ldots M_{k+n}\}, n \in \mathbb{N}$ has a $\glb(S') = M_{k-1}$ and a $\lub(S') = M_{k+n+1}$. For the special case where $S' = S$ we have that $\glb(S) = M_0 = false$ and $\lub(S) = true$.
\end{proof}

\begin{lemma} Algorithm~\ref{alg:synthesis} is a monotonic function on $S$ to $S$.
\label{lm:altmonotonicity}
\end{lemma}
\begin{proof}
Algorithm~\ref{alg:synthesis} recursively reduces the region $S = true$ attempting to reach a greatest fixpoint of viable states $F(s)$, such that the formula $\forall s, i. (F(s) \land A(s,i) \Rightarrow \exists s'. G_T(s,i,s') \land F(s')$ is valid. Notice the difference between $F(s)$ and subsets $M_k$, since the former implies a stronger property over the $\forall\exists$-formula (validity over satisfiability). Initially $F(s) = S = true$, and the algorithm proceeds to compute a new set of viable states. During each iteration, we use $F(s)$ to store this set, and it is the case that for any iteration $i$ of the algorithm, $F_{i+1}(s) \subseteq F_{i}(s)$. As such, we can consider Algorithm~\ref{alg:synthesis} as an isotone, and more specifically increasing function $f$ since for any pair of viable sets $F_{i}, F_{i+1}$, if $F_{i+1} \subseteq F_{i}$, we have that $f(F_{i+1} \subseteq f(F_{i}))$.
\end{proof}

\begin{theorem}[Characterization of Generated Fixpoints]
The set $P$ of all fixpoints in Algorithm~\ref{alg:synthesis} is non
empty, and the system $\tuple{P, \subseteq}$ is a complete lattice.
\label{thm:altfixpoint}
\end{theorem}
\begin{proof}
The proof relies on Tarski's first theorem on fixed
points~\cite{tarski1955lattice}.
Considering Lemmas~\ref{lem:altlattice} and~\ref{lem:altmonotonicity}, we satisfy the first two
conditions of Tarski's theorem. When the specification is realizable, a
fixpoint is reached by Algorithm~\ref{alg:synthesis}, since each consecutive
attempt to further refine $F(s)$ results in the same set. On the other hand, if
the specification is unrealizable, the algorithm returns the fixpoint $F(s) = false$. Therefore, the
set $P$ of all fixpoints in Algorithm~\ref{alg:synthesis} contains at least two
fixpoints. The top-down refinement of the region $S = true$ ensures that the generated fixpoint is always the greatest fixpoint.

Since all three conditions of Tarski's Fixed Point theorem are satisfied by our
solution, we can conclude that $P$ is a non empty set, while the system
$\tuple{P, T}$ is a complete lattice, as it contains a \lub, which is
the solution to a realizable contract, while $\glb = false$, and corresponds to
the solution for an unrealizable contract.
\end{proof}
\fi

%\iffalse
\begin{lemma} Consider the system
$\mathfrak{U} = \tuple{S, T}$. With $S$, we denote the set of
subsets of the orignal state space, such that each subset contains assignments that lead to the
satisfiability of Equation~\ref{eq:viable}. $T$ refers to the
transition relation between any two states, that establishes a partial order
on $S$. Then $\mathfrak{U}$ is a complete lattice, where every subset $B \subseteq
S$ has a greatest lower bound  $\glb = \cap B$  and a least upper
bound $\lub = \cup B$.
\label{lem:lattice}
\end{lemma}
\begin{proof}
Considering the partial order that is established by T, it is straightforward
to show that all subsets $B$ of $S$ contain a \glb and a \lub. These
are respectively, the states which have no preceeding state in $B$ other than
possible ones in the \glb, and the states from which we take a transition into
a new state that's either in the \lub, or outside of $B$. For the special case
where $B = S$, we have that $\glb = false$ and $\lub = true$.
\end{proof}

\john{I really do not follow this proof. Why is it clear that $T$ represents a partial order of states? Is $T$ the result of the synthesis algorithm? Certainly you could have a transition system in which states transition from one back to another and so on and so forth.}


\begin{lemma} Algorithm~\ref{alg:synthesis} is a monotonic function on $S$ to
$S$.

\john{The type of Algorithm~\ref{alg:synthesis}'s input is different from its output so I do not understand how it can map $S$ to $S$}.
\label{lem:monotonicity}
\end{lemma}
\begin{proof}
The algorithm recursively reduces $S$, attempting to reach a fixed point
at which $S$ only contains state assignments that lead to the satisfiability of
Equation~\ref{eq:viable}. As such, it can be considered as an isotone function
$f$, where, for every pair $(B,A)$ with $B \subseteq A \subseteq S$, we have that
$f(B) \subseteq f(A)$.
\end{proof}

\begin{theorem}[Characterization of Generated Fixpoints]
The set $P$ of all fixpoints in Algorithm~\ref{alg:synthesis} is non
empty, and the system $\tuple{P, T}$ is a complete lattice.
\label{thm:fixpoint}
\end{theorem}
\begin{proof}
The proof relies on Tarski's first theorem on fixed
points~\cite{tarski1955lattice}.
Considering Lemmas~\ref{lem:lattice} and~\ref{lem:monotonicity}, we satisfy the first two
conditions of Tarski's theorem. When the specification is realizable, a
fixpoint is reached by Algorithm~\ref{alg:synthesis}, since each consecutive
attempt to further refine $F(s)$ results in the same set. On the other hand, if
the specification is unrealizable, the algorithm returns the fixpoint $F(s) = false$. Therefore, the
set $P$ of all fixpoints in Algorithm~\ref{alg:synthesis} contains at least two
fixpoints.

Since all three conditions of Tarski's Fixed Point theorem are satisfied by our
solution, we can conclude that $P$ is a non empty set, while the system
$\tuple{P, T}$ is a complete lattice, as it contains a \lub, which is
the solution to a realizable contract, while $\glb = false$, and corresponds to
the solution for an unrealizable contract.
\end{proof}
%\fi

From Theorem~\ref{thm:fixpoint} we have shown that Algorithm~\ref{alg:synthesis} computes a fixpoint for both cases where the specification is realizable or not. With this knowledge at hand, it is necessary to prove the soundness of the fixpoints generated. In other words, when a fixpoint is computed for a realizable contract, it should be the case that the algorithm reports a ``realizable'' result. The same needs to hold for the dual case of ``unrealizable'' results.

\begin{theorem}[Soundness of ``realizable'' results]
\label{thm:sndreal}

Assume a sound quantifier elimination process that provides us with exact regions of validity. If $F(s)$ is a fixpoint generated by Algorithm~\ref{alg:synthesis} and $F(s) \neq false$, then the contract is realizable.
\end{theorem}
\begin{proof} To prove this theorem, we show that $\forall s. \viable(s) \Rightarrow F(s)$ using induction on $F(s)$. The base case is covered since $F(s) \land G_I(s) \neq \varnothing$. For the inductive case, we require that a state $x$ exists, such that $G_T(s,i,x) \land F(x)$. Since $\viable(s)$ is true, we know that $\exists s'. G_T(s,i,s') \land \viable(s')$. Thus we pick $s'$ and use the inductive hypothesis to show that $\viable(s') \Rightarrow F(s')$.
\end{proof}

\john{The base case of this proof does not imply that $\exists s. \viable(s) \Rightarrow F(s)$. Viability is not defined in terms of $G_I(s)$. Also, even if you can show the base case is satisfied you need to make an assumption that $A(s,i)$ is non-empty in order to have $\exists s'. G_T(s,i,s') \land \viable(s')$. Also to imply induction this way don't we need to show that $\forall_i. G_T(s,i,s')$ is some sort of well-founded relation over $S$?}

\begin{theorem}[Soundness of ``unrealizable" results]
\label{thm:sndunreal}

Assume a quantifier elimination process that provides exact regions of validity. If the output of Algorithm~\ref{alg:synthesis} is the fixpoint $F(s) = false$, then the contract is unrealizable.
\end{theorem}
\begin{proof}
Dually to Theorem~\ref{thm:sndreal} we show that $\forall s. \lnot \viable(s) \Rightarrow \lnot F(s)$ using induction on $F(s)$. From this point on, the proof direction is analogous to that of Theorem~\ref{thm:sndreal}.
\end{proof}


\begin{corollary}[Soundness of Realizability results from Validity-Guided Synthesis]
Assume a quantifier elimination process that provides exact regions of validity. The process described in Algorithm~\ref{alg:synthesis} is sound.
\end{corollary}

Notice how the Theorems on the soundness of the results provided by the validity-guided approach assume a quantifier elimination process that computes exact regions of validity. This is a crucial requirement, as it determines the overall effectiveness of the approach. The algorithm is still applicable to cases where the assumption is not met, however its effectiveness on providing sound results is directly affected. Not considering the case of exact regions, we have two other cases, overapproximations and underapproximations. In the former case, the regions of unsafe states (i.e. the negation of a region of validity) would be underapproximations and the algorithm's performance decreases, since problems might not be solvable as fast as with the use of exact regions. On the other hand, if the regions of validity are underapproximations, their negations are overapproximations, and as such, the blocked regions might cover safe states in their constraints. This makes the algorithm follow a more pessimistic approach, where it overconstrains the problem to find a solution. This might lead to cases where an otherwise realizable contract might be declared as unrealizable by the process. We show how the latter can be manifested in practice in Section~\ref{sec:results}, where we compare our approach to a synthesis algorithm that is based on k-induction and is prone to unsound ``unrealizable'' results. In the context of this paper, and considering Lemma~\ref{lem:subset}, \aeval is able to provide exact regions of validity.

\subsection{Applying \jsynvg to the Cinderella-Stepmother game}
\label{sec:algexample}

\begin{figure}[!t]
\centering
 \begin{Verbatim}[fontsize=\footnotesize]
const C = 2.0;

-- empty buckets e and e+1 each round
node game(i1,i2,i3,i4,i5: real; e: int)
	returns (guarantee: bool);
var
  b1, b2, b3, b4, b5 : real;
let
  assert i1 >= 0.0 and i2 >= 0.0 and
 	i3 >= 0.0 and i4 >= 0.0 and i5 >= 0.0;
  assert i1 + i2 + i3 + i4 + i5 = 1.0;

  b1 = 0.0 ->
       (if (e = 5 or e = 1) then i1
                            else (pre(b1) + i1));
  b2 = 0.0 ->
       (if (e = 1 or e = 2) then i2
                            else (pre(b2) + i2));
  b3 = 0.0 ->
       (if (e = 2 or e = 3) then i3
                            else (pre(b3) + i3));
  b4 = 0.0 ->
       (if (e = 3 or e = 4) then i4
                            else (pre(b4) + i4));
  b5 = 0.0 ->
       (if (e = 4 or e = 5) then i5
                            else (pre(b5) + i5));

  guarantee = b1 <= C and b2 <= C and b3 <= C and
  	    b4 <= C and b5 <= C;

  --%REALIZABLE i1, i2, i3, i4, i5;
  --%PROPERTY guarantee;
tel;
 \end{Verbatim}
\caption{An Assume-Guarantee contract for the Cinderella-Stepmother game in Lustre}
\label{fg:cind}
\end{figure}

Figure~\ref{fg:cind} shows one possible interpretation of the contract designed
for the instance of the Cinderella-Stepmother game that we introduced in Section~\ref{sec:example}. The contract
is expressed using Lustre~\cite{lustrev6}, a language
that has been extensively used for specification as well as implementation of
safety-critical systems, and is the kernel language in SCADE, a popular tool in
model-based development. The contract is defined as a Lustre node, with a global
constant $C$ denoting the bucket capacity. The node describes the game itself,
through the problem's input and output variables. The main input is Stepmother's
distribution of one unit of water over five different input variables,
\textit{i1} to \textit{i5}. While the node contains a sixth input argument,
namely $e$, this is in fact used as the output of the system that we want to
implement, representing Cinderella's choice at each of her turns. We make an
explicit distinction on which node arguments are the system's inputs, using the \textit{REALIZABLE} statement towards the end of the contract. The contract's
assumptions $A$ are defined using assertions over the input variables
\textit{i1-i5}. We assume that there can be no negative input values, and the
sum of \textit{i1} to \textit{i5} is equal to one unit of water. The specified node
returns the value of a boolean variable \textit{guarantee}, which corresponds to
the contract's guarantee $G$, and is explicitly defined to be such using the
\textit{PROPERTY} statement.
Finally, we define what each bucket's state should be through the entire
duration of the game, using the local variables \textit{b1} to \textit{b5}.
Initially, each bucket is empty, and with each transition to a new state, the contents depend on
whether Cinderella chose the specific bucket, or an adjacent one. If this is the
case, the contents of the bucket $b_i$ become equal to the amount of water that
Stepmother puts in the corresponding input variable $i$, during the next turn.
If the bucket was not covered by Cinderella's choice, then its contents are
updated by adding Stepmother's distribution to the volume of water that the
bucket already had. The distinction between the initial state of each bucket,
and every other state is expressed using Lustre's arrow (\texttt{->})
operator, while variable values in previous states can be accessed using the
\textit{pre} operation.
The specification is fairly simple, as the objective is clear. The contract
should only be realizable if, assuming valid inputs given by the Stepmother
(i.e. positive values to input variables that add up to one water unit),
Cinderella can keep reacting indefinitely, by providing outputs that satisfy the
guarantees (i.e. she empties buckets in order to prevent overflow in Stepmother's next turn).
We initiate our call to the procedure defined in Algorithm~\ref{alg:synthesis},
by providing the contract in Figure~\ref{fg:cind} as input. The algorithm
then enters an infinite loop, where it attempts to construct a fixpoint of
viable states that can use the transition relation to new, viable states,
while complying to the specification. Initially $F(s) = true$, and we ask \aeval for the validiity of the formula $\phi = \forall s,i. \ (F(s) \land A(s,i) \Rightarrow \exists s'.G_{T}(s,i,s') \land F(s'))$. Unfortunately, not every state in $F(s)$
satisfies the formula. As such $\phi$ is not valid, and \aeval provides us with
a region of validity $Q(s,i)$, which is a subset of $F(s) \land A(s,i)$ for
which the formula $\phi$ is valid. Since $Q(s,i)$ is defined over both state and
input variables, it might contain contraints over the inputs, which is an
undesirable side-effect. Due to space restrictions, we are unable to show this
in full effect, but Figure~\ref{fg:snippet} shows part of the generated
$Q(s,i)$, in SMT-LIB format (pre-fix notation), where the input variables are
part of the constraints. The variable \textit{\%init} is used in the underlying
machinery as a flag to indicate whether the current state is initial or not
(true and false, respectively). \andrew{Is there anything else we can
  say about the code in the figure? Does it give us any insight into a
  possible solution for cinderella?}

\begin{figure}[!t]
\centering
 \begin{Verbatim}[fontsize=\footnotesize]
(and (or (and %init (= (+ b1 i1) 0.0)) (not %init))
  (or (and %init (= (+ b2 i2) 0.0)) (not %init))
  (or (and %init (= (+ b3 i3) 0.0)) (not %init))
  (or (and %init (= i4 0.0)) (not %init))
  (or (and %init (= i5 0.0)) (not %init))
  (<= (+ b1 i1) 2.0)
  (<= (+ b2 i2) 2.0)
  (<= (+ b3 i3) 2.0)
  (<= i4 2.0)
  (<= i5 2.0))
 \end{Verbatim}
\caption{Code snippet of the region of validity generated for the Cinderella-Stepmother
example}
\label{fg:snippet}
\end{figure}

Considering the existence of undesirable constraints in $Q(s,i)$, the next step
is to extract a region of validity that is only described using state variables.
As such, we construct the formula $\phi' = \forall s. \ (F(s) \Rightarrow \exists
i. A(s,i) \land \lnot Q(s,i))$, and ask \aeval regarding its validity. According
to \aeval, $\phi'$ is not valid, an indication that may eventually lead us to a
proof of the contract's realizability. If $\phi'$ was valid, then every state in
$F(s)$ is unsafe, under a specific input that satisfies the contract
assumptions (since $\lnot Q(s,i)$ holds in this case), and the specification is unrealizable. Since this is not the case,
however, \aeval computes a region of validity $W(s)$ that describes states that are unsafe under certain valid inputs. As a consequence, a refinement process comes at the next step. in which we derive $F(s) = F(s) \land \lnot W(s)$, thus blocking the unsafe region $W(s)$.

From this point on, the algorithm reiterates the process following the steps
explained above. For this particular example, the algorithm terminates after one
more refinement, at depth 2. At that point, the refined version of
$\phi$ is valid, and \aeval constructs a witness containing valid reactions to
environment behavior. Figure~\ref{fg:witness} provides a snippet of
that witness, after being translated to a C implementation. \andrew{What's with the 'if' in the middle of the 'else' block? Is it
  suppose to be an 'else-if'?}
In general,
the witness is described through the use of an nested if-then-else block, where the conditions are subsets of the antecedent of
the implication in formula $\phi$, while the body contains valid assignments to
state variables, for this particular subset.


\begin{figure}[!t]
\centering
 \begin{Verbatim}[fontsize=\footnotesize]
...
if (!(b1 + i1 <= 0.0) || !((b2 + i2) <= 0.0) || ...)
  e = 3;
  b1 = b1 + i1;
  b2 = b2 + i2;
  b3 = i3;
  b4 = i4;
  b5 = b5 + i5;
} else {
  if (!(i1 <= 0.0) || !((b2 + i2) <= 0.0) || ...)
  e = 5;
  b1 = i1;
  b2 = b2 + i2;
  b3 = b3 + i3;
  b4 = b4 + i4;
  b5 = i5;
} else {
...
 \end{Verbatim}
\caption{Code snippet of the synthesized implementation for the Cinderella-Stepmother
example}
\label{fg:witness}
\end{figure}

As is discussed by Beyene \textit{et al.}~\cite{beyene2014constraint}, the
Cinderella-Stepmother is a rather challenging game. In particular, for the
cases where $1.5 \leq C \leq 3.0$, the problem becomes non-trivial, and their
work showed how familiarity with the problem can help alleviate this complexity
through the use of pre-defined templates. Despite this, we were still
able to synthesize implementations for the Cinderella-Stepmother game for any
value of $C \geq 2$, using a completely automatic approach.~\footnote{The region $1.5\leq C < 2$ is a particularly hard problem and the algorithm does not converge with a timeout up to seven hours.} This fact alone,
provided us with solid foundations regarding the algorithm's effectiveness in
solving complex problems.
