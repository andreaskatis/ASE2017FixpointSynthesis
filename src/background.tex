\section{Background}
\label{sec:background}


In this section we define Assume-Guarantee contracts (Section~\ref{sec:pre}),
describe the problem of synthesis in terms of discovering inductive invariants that imply the realizability of the given specification (Section~\ref{sec:formals}), and show how a sophisticated solver is used in order to compute these fixpoints (Section~\ref{sec:aeval}).
%Finally, we enrich our formal definitions with an informal proof of the
%algorithm's correctness in terms of the successfully synthesized
%implementations.

\subsection{Assume-Guarantee Contracts}
\label{sec:pre}

For the purposes of this paper, we focus our interest in a mainstream variation
for representing system requirements, using an \textit{Assume-Guarantee
Contract}. Requirements in this format contain two main types of constraints.
The \emph{assumptions} of the contract restrict the possible inputs that the
environment can provide to the system, while the \emph{guarantees} are used to
describe what is considered a safe reaction of the system to the outside world.

A simple example is the contract with assumption $A = \{x\neq
y\}$ and guarantee $G = \{x \leq y \Rightarrow z =
\textit{true}, x \geq y \Rightarrow z = \textit{false}\}$. Variables
$x$ and $y$ are the designated inputs while $z$ is the output. This is a
well-defined contract, since by the assumption $A$, it is always the case that
$x \neq y$, therefore at least one implementation exists, which, for example
sets $z$ to true if $x < y$ and false otherwise. An alternative
valid implementation could set $z$ to false if $x > y$, and true otherwise. The
proof of existence of such an implementation is the main concept behind the
\emph{realizability} problem, while the automated construction of a witness
implementation is the main focus of \emph{program synthesis}.


It is apparent that the example contract above is therefore \emph{realizable},
and an efficient synthesis procedure would be capable of providing at least one
implementation. Nevertheless, it is important to consider a variation of the
example, where $A = \varnothing$. This is a practical case of an
\emph{unrealizable} contract, as there is no feasible implementation that can
correctly react to the environment assigning values to $x$ and $y$, such that
$x = y$.

\subsection{Formal Representation}
\label{sec:formals}
We use two disjoint sets, $state$ and $inputs$, to describe a system.
A straightforward and intuitive way to represent an \emph{implementation} is by
defining a \emph{transition system}, composed of an initial state
predicate $I(s)$ of type $state \to bool$, as well as a transition relation
$T(s,i,s')$ of type $state \to inputs \to state \to bool$.

Combining the above, we represent an Assume-Guarantee (AG) contract using a set
of \emph{assumptions}, $A: state \rightarrow inputs \rightarrow bool$,
and a set of \emph{guarantees} $G$. The latter is further decomposed into two
distinct subsets $G_I: state \rightarrow bool$ and $G_T: state \rightarrow
inputs \rightarrow state \rightarrow bool$. $G_I$ defines the set of valid
initial states, and $G_T$ contains constraints that need to be satisfied in
every transition between two states. An important note at this point is that we
we do not make any distinction between internal state variables and outputs in the
formalism. This alone allows us to use state variables to (in some cases)
simplify specification of guarantees, since we do not expect a contract
to be always defined over all variables in the transition system.

Consequently, we can formally define a realizable contract, as one for which any
preceeding state $s$ can take a transition into a new state $s'$ that satisfies
the guarantees, assuming valid inputs. For a system to be ever-reactive, these
new states $s'$ should be further usable as preceeding states in a future
transition. States like $s$ and $s'$ are defined as being \textit{viable}, if
and only if:

\begin{equation}
\forall s,i.~ A(s, i) \land \viable(s) \Rightarrow \exists s'.~ G_T(s, i,s')
\land \viable(s')
\label{eq:viable}
\end{equation}

A necessary condition, finally, is that the set of viable states
intersects with the set of initial states. As such, to conclude that a contract
is realizable, we require that

\begin{equation}
\viable(s) \land G_I(s) \neq \varnothing
\label{eq:nonempty}
\end{equation}

The intuition behind our proposed algorithm in this paper relies on the
discovery of a greatest fixpoint that only contains viable states. In the case where a fixpoint is computed, we proceed by extract a witnessing collection of reactions that are, by construction, guaranteed to satisfy the specification. To achieve both the fixpoint generation, as well as the witness extraction, we depend on \aeval, a sophisticated solver for $\forall\exists$-formulas.

\subsection{\textit{AE-VAL}}
\label{sec:aeval}

\begin{figure}[!t]
\centering
\includegraphics[width=3.5in]{aeval_invalid}
\caption{Region of validity computed for an example requiring \aeval to iterate two times.}
\label{fg:aeval}
\end{figure}

\andreas{I believe the section can be a bit longer. I also think that the keyphrase ``region of validity'' needs to stand out more. Finally, a proof for Lemma 1, or at least an outline of it would be greatly appreciated.}

\aeval~\cite{fedyukovich2015automated} is an algorithm to decide validity and extract Skolem functions.
It takes as input a formula of the form $\forall s \,.\,  A(s) \Rightarrow \exists s' . G(s,s')$, 
where $A(s)$ has only existential\andreas{you meant universal here, right?} quantifiers, and $G(s,s')$ is quantifier-free.
%
\andreas{I suppose that the partitions P are the ones named T in the Figure. I think we should keep the same name or the reader might get confused.}
While deciding the validity, \aeval iteratively enumerates models of $A(s) \land G (s, s')$ and groups them into a set of partitions $\{P_i(s)\}$, such that each $P_i(s) \Rightarrow \exists s' . G (s, s')$.
We say that after $n$ iterations, \aeval establishes a formula $R_n(s) \eqdef \bigvee_{i=1}^n P_i(s)$ which is by definition an under-approximation of $\exists s' . G (s, s')$.

If after $n$ iterations, it happens that $A(s) \Rightarrow R_n(s)$ then $\forall s \,.\,  A(s) \Rightarrow \exists s' . G(s,s')$ is valid, and \aeval generates a Skolem function as described in~\cite{katis2016synthesis}.
Alternatively, if $A(s) \land  G (s, s') \land \neg{R_n (s, s')}$ is unsatisfiable, then $A(s) \land \neg G (s, s')$ is satisfiable, or equivalently $\forall s \,.\,  A(s) \Rightarrow \exists s' . G(s,s')$ is invalid (see an example in Figure~\ref{fg:aeval}).
In both cases, we say that $A(s) \land R_n(s)$ is a \emph{region of validity}, meaning that $\forall s \,.\,  A(s) \land R_n(s) \Rightarrow \exists s' . G(s,s')$ is valid by construction.

\begin{lemma}
If formula $\forall s \,.\,  A(s) \Rightarrow \exists s' . G(s,s')$ is invalid, and $A(s) \land R_n(s)$ is the region of validity, then there is no other formula $S(s)$ such that $A(s) \land R_n(s) \Rightarrow S(s)$ and $\forall s \,.\,  S(s) \Rightarrow \exists s' . G(s,s')$.

\label{lem:subset}
\end{lemma}

