\section{Conclusion and Future Work}
\label{sec:conclusion}

In this paper, we contributed an approach to program synthesis guided by the
proofs of realizability of Assume-Guarantee contracts.
To check realizability, it performs k-induction-based reasoning to decide validity of a set of $\forall\exists$-formulas.
Whenever a contract is proven realizable, it further employs the Skolemization procedure and extracts a fine-grained witness to the realizability.
Those Skolem functions are finally encoded into a desirable implementation.
We implemented the technique in the \jsynvg tool and evaluated it for the set
of Lustre models of different complexity.
The experimental results provided fruitful conclusions on the overall efficacy of the the approach.

To the best of our knowledge our work is the first complete attempt on providing
a synthesis algorithm based on the principle of k-induction using infinite
theories. The ability to express contracts that support ideas from many
categories of specifications, such as template-based and temporal properties,
increases the potential applicability of this work to multiple subareas on
synthesis research.

In future work, we plan to exploit the connections of our approach with
%eventually switch to a more efficient algorithm, where we endorse the core
Property Directed Reachability~\cite{komuravelli2014smt} more closely. %~\cite{een2011efficient,bradley11}.
%using efficient ways to further enhance the algorithm's performance through the
%use of implicit abstractions~\cite{cimatti2014ic3} to further reduce the search
%space of the algorithm.
%This will also help our original work on realizability checking, by improving the unsoundness of our unrealizable results. 
Another promising idea here is the use of Inductive Validity Cores (IVCs)~\cite{Ghass16}, whose main purpose is to effectively pinpoint the absolutely necessary model elements in a generated proof. We can potentially use
the information provided by IVCs as a preprocessing tool to reduce the size of
the original specification, and hopefully the complexity of the realizability
proof. Finally, various optimizations can be implemented on top of both
\aeval and \textsc{SMTLib2C} to produce smaller implementations.