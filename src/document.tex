%%This is a very basic article template.
%%There is just one section and two subsections.
\documentclass[10pt,conference]{llncs}

%to handle ranges of citations in IEEEtran
\usepackage[noadjust]{cite}
\renewcommand{\citepunct}{,\penalty\citepunctpenalty\,}
\renewcommand{\citedash}{--}% optionally

%To fix ``fi'' encoding
\usepackage{cmap}
%\usepackage{amsthm}
\usepackage{amsmath}
\usepackage{listings}
\usepackage{amsfonts}
\usepackage{graphicx}
\usepackage{courier}
\usepackage{algorithm}
\usepackage[noend]{algpseudocode}
%--- Return lines start in a new line
\let\oldReturn\Return
\renewcommand{\Return}{\State\oldReturn}
%---
\usepackage[table,xcdraw]{xcolor}
\usepackage{float}
\usepackage{hyperref}
\usepackage{mathtools}
\usepackage[framemethod=TikZ]{mdframed}
\usepackage[]{inputenc}
\usepackage[T1]{fontenc}
\usepackage{placeins}
\usepackage{amsmath,amssymb}
\usepackage{xspace}
\usepackage[font=footnotesize]{caption}
\captionsetup[table]{skip=10pt}
%to solve the hyperref problem of jumping to wrong places
%\usepackage[all]{hypcap}
\usepackage[framemethod=TikZ]{mdframed}
\usepackage{booktabs}
\usepackage{fancyvrb}
\usepackage{relsize}
\graphicspath{{images/}}
%\usepackage[ruled,shortend,linesnumbered,algo2e]{algorithm2e}  % algo2e = use
% \begin{algorithm2e}
\usepackage{float}
\usepackage{subfig}
\usepackage{framed}
\usepackage{multirow}

\usepackage[firstpage]{draftwatermark}
\SetWatermarkText{\hspace*{6in}\raisebox{6.3in}{\includegraphics[scale=0.1]{aec-badge-tacas}}}
\SetWatermarkAngle{0}
%\newtheorem{theorem}{Theorem}
%\newtheorem{lemma}{Lemma}
%\newtheorem{corollary}{Corollary}

\newcommand{\aeval}{\textsc{AE-VAL}\xspace}
\newcommand{\jkind}{\textsc{JKind}\xspace}

\newcommand{\jsyn}{\textsc{JSyn}\xspace}

\newcommand{\jsynvg}{\textsc{JSyn-vg}\xspace}
% \newcommand{\jsynvg}{\textsc{Jvgs}\xspace}
\newcommand{\smtlibtoc}{\textsc{SMTLib2C}\xspace}
\newcommand{\lustrev}{\textsc{LustreV6}\xspace}

\newcommand{\isSat}{\textsc{isSat}\xspace}
\newcommand{\isUnSat}{\textsc{isUnsat}\xspace}

\newcommand{\viable}{{\mathsf {Viable}}}
\newcommand{\reachable}{{\mathsf {Reachable}}}

\newcommand{\extend}{{\mathsf {Extend}}}
\newcommand{\basecheck}{{\mathsf {BaseCheck}}}
\newcommand{\extendcheck}{{\mathsf {ExtendCheck}}}

\newcommand{\glb}{\textit {GLB}\xspace}
\newcommand{\lub}{\textit {LUB}\xspace}
\newcommand{\tuple}[1]{\langle #1 \rangle}

\renewcommand{\labelitemi}{\tiny$\blacksquare$}

\newcommand{\andreas}[1]{\textcolor{blue}{Andreas: #1}}
\newcommand{\mike}[1]{\textcolor{red}{Mike: #1}}
\newcommand{\andrew}[1]{\textcolor{green}{Andrew: #1}}
\newcommand{\john}[1]{\textcolor{orange}{John: #1}}
\newcommand{\grigory}[1]{\textcolor{brown}{Grigory: #1}}
\newcommand{\arie}[1]{\textcolor{purple}{[Arie: #1]}}
\newcommand{\huajun}[1]{\textcolor{yellow}{[Huajun: #1]}}

\newcommand{\realizable}{\textsc{realizable}\xspace}
\newcommand{\unrealizable}{\textsc{unrealizable}\xspace}
\newcommand{\skolems}{\textit{Skolem}}
\newcommand{\aevalres}{\textit{aevalResult}}
\newcommand{\init}{\textit{init}}
\newcommand{\subs}{\textit{validRegion}}
\newcommand{\isValid}{\textsc{isValid}\xspace}
\newcommand{\isInvalid}{\textsc{isInvalid}\xspace}
%\newcommand{\isSat}{\textsc{isSat}\xspace}
\newcommand{\isUnsat}{\textsc{isUnsat}\xspace}

\newcommand\eqdef{\mathrel{\stackrel{\makebox[0pt]{\mbox{\normalfont\tiny def}}}{=}}}

\newcounter{template}
\newenvironment{template}[1][htb]
  {
   \begin{algorithm2e}[#1]%
   \SetAlgorithmName{Template}
  }{\end{algorithm2e}}

\newenvironment{requirement}
{\vspace{0.05in}
 \begin{mdframed}[roundcorner=10pt,backgroundcolor=gray!20]}
{\end{mdframed}}


\begin{document}
\title{Validity-Guided Synthesis of Reactive Systems from Assume-Guarantee Contracts}

\iffalse
\author{\IEEEauthorblockN{Andreas Katis\IEEEauthorrefmark{1}, Grigory Fedyukovich\IEEEauthorrefmark{2},  Huajun
Guo\IEEEauthorrefmark{1}, Andrew Gacek\IEEEauthorrefmark{3}\\ John
Backes\IEEEauthorrefmark{3}, Arie Gurfinkel\IEEEauthorrefmark{4} and
Michael W. Whalen\IEEEauthorrefmark{1}}
\IEEEauthorblockA{\IEEEauthorrefmark{1}Department of Computer Science and
Engineering,
University of Minnesota\\
Email: \{katis001,guoxx663\}@umn.edu, whalen@cs.umn.edu}
\IEEEauthorblockA{\IEEEauthorrefmark{2}University of Washington Paul G. Allen School of Computer Science \& Engineering\\
Email: grigory@cs.washington.edu}
\IEEEauthorblockA{\IEEEauthorrefmark{3}Rockwell Collins Advanced Technology Center\\
Email: \{andrew.gacek,john.backes\}@rockwellcollins.com}
\IEEEauthorblockA{\IEEEauthorrefmark{4}Department of Electrical and Computer
Engineering, University of Waterloo\\
Email: agurfinkel@uwaterloo.ca}}
\fi 

\author{Andreas Katis\inst{1}, Grigory Fedyukovich\inst{2}, Huajun Guo\inst{1}, 
  Andrew Gacek\inst{3}, John Backes\inst{3}, Arie Gurfinkel\inst{4}, Michael
  W. Whalen\inst{1}}%

\institute{
Department of Computer Science and Engineering, University of Minnesota\\
\email{\{katis001,guoxx663\}@umn.edu, whalen@cs.umn.edu}
\and
Department of Computer Science, Princeton University\\
\email{grigoryf@cs.princeton.edu}
\and
Rockwell Collins Advanced Technology Center\\
\email{\{andrew.gacek,john.backes\}@rockwellcollins.com}
\and
Department of Electrical and Computer Engineering, University of Waterloo\\
\email{agurfinkel@uwaterloo.ca}
}

%\author{\IEEEauthorblockN{Authors anonymized for submission to ASE}}


\maketitle

\begin{abstract}

%shortened version due to 200 word limit
Automated synthesis of reactive systems from specifications has been a topic of research for decades.  Recently, a variety of approaches have been proposed to extend synthesis of reactive systems from propositional specifications towards specifications over rich theories. 
%Such approaches include inductive synthesis, template-based synthesis, counterexample-guided synthesis, and predicate abstraction techniques. In this paper, w
We propose a novel, completely automated approach to program synthesis which reduces the problem to deciding the validity of a set of $\forall\exists$-formulas. 
In spirit of 
%Inspired by verification techniques that construct inductive invariants, like 
IC3 / PDR, our problem space is recursively refined by blocking out regions of unsafe states, aiming to discover a fixpoint that describes safe reactions.
If such a fixpoint is found, we construct a witness that is directly translated into an implementation. 
We implemented the algorithm on top of the \jkind model checker, and exercised it against contracts written using
the Lustre specification language. 
Experimental results show how the new algorithm outperforms 
\jkind's already existing synthesis procedure based on k-induction and addresses soundness issues in the k-inductive approach with respect to unrealizable results.
%\grigory{did not follow the  message about ``soundness for unrealizable results'' and dropped it for a while. Perhaps you could rephrase it and possibly move towards the beginning of the abstract.}
%Experimental results show how the new algorithm yields better performance as well as soundness for ``unrealizable'' results when compared to \jkind's existing synthesis procedure, an approach based on the construction of k-inductive proofs of realizability.

\iffalse
Automated synthesis of reactive systems using only specification knowledge is one of
the most popular and well explored subjects in formal verification. A variety of
approaches have been proposed in the recent years, through inductive and
functional synthesis, while template-based, counterexample-guided and predicate abstraction techniques
have been proved useful towards extending the applicability of such algorithms
to solving difficult problems. In this paper, we propose a novel approach to
program synthesis, which is based on the validity of a $\forall\exists$-formula.
The approach is inspired from verification techniques that construct
inductive invariants, like Property Directed Reachability, and is completely automated. The original
problem space is recursively refined by blocking out regions of unsafe states, with the goal being the discovery of a
fixpont that describes a region containing safe reactions. If such a fixpoint
is found, we construct a witness that can be directly transformed into an
implementation using mainstream programming languages, like C. We implemented
the algorithm in the \jkind model checker, and exercised it against contracts
written using the Lustre specification language. Experimental results show how
the new algorithm yields better performance as well as soundness for ``unrealizable'' results when compared to \jkind's existing synthesis procedure, an approach that is based on the construction of k-inductive proofs of realizability.
\fi
\end{abstract}


\section{Introduction}
\andreas{The following paragraph is the same with the one in the ATVA paper}
Automated synthesis research is concerned with discovering efficient algorithms to construct programs that are guaranteed to comply with predefined temporal specifications.
This problem has been well studied for propositional specifications, especially for (subsets of) LTL~\cite{gulwani2010dimensions}.
More recently, the problem of synthesizing programs for richer theories has been
examined, including work in {\em template-based
synthesis}~\cite{srivastava2013template}, which attempts to find programs that
match a certain shape (the template), and {\em inductive synthesis}, which
attempts to use counterexample-based refinement to solve synthesis problems~\cite{flener2001inductive}.  Such techniques have been widely used for stateless formulas over arithmetic domains~\cite{reynolds2015counterexample}.
\textit{Functional synthesis} has also been effectively used to synthesize
subcomponents of already existing partial
implementations~\cite{kuncak2013functional}.

In this work, we describe a novel approach that can effectively synthesize
programs using specifications written in the form of an arbitrary {\em
assume/guarantee contract}. The technique has been excersised to specification
that includes safety properties in linear real and integer arithmetic (LIRA),
but remains generic enough to be extended into supporting additional theories
in the future. This is a ``hands-off'' approach, as there is no
requirement from the user to interact with the underlying machinery,
unlike~\cite{ryzhyk2014user,ryzhyk2016developing}, and is capable of providing
solutions without the guidance of templates, like
in~\cite{beyene2014constraint}.

The main idea of the algorithm was inspired by the IC3 algorithm, its
technique otherwise known as Property Directed Reachability
(PDR)~\cite{bradley2011sat,een2011efficient}. In PDR, one aims to discover an
inductive invariant for a property, by recursively blocking generalized regions describing unsafe states. In a similar concept, we attempt
to reach a greatest fixpoint that contains states usable indefinitely by the
system, in order to react to unpredictable environment behaviour, while
complying to well-defined specification. As such, beginning from the entire
problem space, we recursively block regions of states that violate the contract, using \textit{regions of validity} that are
generated by non-valid $\forall\exists$ formulas. If the refined
$\forall\exists$ formula is valid, we reach an approximation of a fixpoint which can effectively be used by the specified transition relation, to
provide safe reactions to environment inputs. We then extract a witness for the
formula's satisfiability, which can be directly transformed into the
language intended for the system's implementation.

The algorithm was implemented as a feature to the \jkind model checker, which
already had unofficial support to program synthesis according to work based on
k-inductive proofs of a contract's
realizability~\cite{gacek2015towards,katis2016towards,katis2016synthesis}.
Our approach is completely independent of the k-inductive method, but is still
based on the extraction of a witness that satisfies a $\forall\exists$ formula, using the \aeval Skolemizer~\cite{fedyukovich2015automated}. To extract such a
witness, we do not depend on a k-inductive proof, but instead use the
\textit{regions of validity} that \aeval can generate from non-valid formulas,
to reach a fixpoint of satisfiable assignments to state variables.
This approach is a direct improvement over the k-inductive method in two
important aspects; performance, and soundness of 'unrealizable' results. While
the former is self-explanatory, with the latter we refer to cases where the
k-inductive algorithm would spuriously report a contract as 'unrealizable', when a correct
implementation actually exists. This unsoundness stems from the pessimistic
behavior of the k-inductive algorithm, as it is not capable to only consider a
``safe'' subset of the state space. We were able to confirm our claims
by comparing the two algorithms under a comprehensive benchmark suite containing
contracts that were initially used in verification problems, as well as
specification for industrial-level designs.

In Section~\ref{sec:synthesis} we provide the background that is required in the
context of this paper, and present the validity-guided approach to synthesizing
implementations. Section~\ref{sec:aeval} contains information regarding \aeval's
capability to compute valid subsets from non-valid $\forall\exists$ formulas.
An outline of the algorithm's implementation is described
in Section~\ref{sec:impl}, and we show the advancements of our validity-guided
approach against the k-inductive method in Section~\ref{sec:results}.
We discuss the differences of our work with closely related approaches In
Section~\ref{sec:related}. Finally, we discuss potential future work and
conclude in Section~\ref{sec:conclusion}.


\section{Running Example: The Cinderella-Stepmother Game}
\label{sec:example}

For the purposes of this paper, we will illustrate how the validity-guided synthesis algorithm works, using a variation of the minimum-backlog
problem, the two player game between Cinderella and her wicked
Stepmother, first  expressed by Bodlaender \textit{et
al.}~\cite{bodlaender2012cinderella}.

The main objective for Cinderella (i.e. the reactive system) is to prevent a
collection of buckets from overflowing with water. On the other hand,
Cinderella's Stepmother (i.e. the system's environment) refills the buckets with a predifined amount of water that is distributed in a random fashion between the buckets.
For the running example, we chose an instance of the game that has been
previously used in template-based synthesis~\cite{beyene2014constraint}. In this instance, the game is described
using five buckets, where each bucket can contain up to two units of water.
Cinderella has the option to empty two adjacent buckets at each of her turns,
while the Stepmother distributes one unit of water over all five buckets. In the context of this paper we use this example to show how specification is expressed, as well as how we can synthesize an efficient implementation that describes reactions for Cinderella, such that an bucket overflow is always prevented.



\section{Background}
\label{sec:background}

\begin{figure}[!t]
\centering
\includegraphics[width=3.5in]{agcontract-crop.pdf}
\caption{An Assume-Guarantee Contract.}
\label{fg:agcontract}
\end{figure}


In this section we define Assume-Guarantee contracts (Section~\ref{sec:pre}),
describe the problem of synthesis in terms of discovering inductive invariants that imply the realizability of the given specification (Section~\ref{sec:formals}), and give a background on our main solving engine (Section~\ref{sec:aeval}).
%Finally, we enrich our formal definitions with an informal proof of the
%algorithm's correctness in terms of the successfully synthesized
%implementations.

\subsection{Assume-Guarantee Contracts}
\label{sec:pre}

We focus our interest on a mainstream variation for representing system requirements, using an \textit{Assume-Guarantee
Contract}. Requirements in this format contain two main types of constraints.
The \emph{assumptions} of the contract restrict the possible inputs that the
environment can provide to the system, while the \emph{guarantees} are used to
describe what is considered a safe reaction of the system to the outside world.

A (conceptually) simple example is shown in Figure~\ref{fg:agcontract}. The contract describes a possible set of requirements for a specific instance of the Cinderella-Stepmother game that we introduced in Section~\ref{sec:example}. Our goal is to synthesize an implementation that describes Cinderella's winning region of the game. Cinderella in this case is the implementation, as shown by the middle box in Figure~\ref{fg:agcontract}. Cinderella's inputs are five different values $i_k, 1 \leq k \leq 5$, determined by a random distribution of one unit of water by the Stepmother. During each of her turns Cinderella has to make a choice denoted by the output variable $e$, such that the buckets $b_k$ do not overflow during the next action of her Stepmother. We define the contract using the set of assumptions $A = \{i_k \geq 0, \sum_{k=1}^{5} i_k = 1\}$ and the guarantees $G = \{b_k \leq 2, b_k = ite((e=k \lor e=k+1), 0, (b_k+i_k))\}$. For the particular example, it is possible to construct at least one implementation that satisfies the guarantee constraints, given the input assumptions. The proof of existence of such an implementation  is the main concept behind the \emph{realizability} problem, while the automated construction of a witness implementation is the main focus of \emph{program synthesis}.


Since the contract in Figure~\ref{fg:agcontract} is \emph{realizable}, an efficient synthesis procedure would be capable of providing at least one
implementation. At this point it is important to consider a variation of the example, where $A = \varnothing$. This is a practical case of an
\emph{unrealizable} contract, as there is no feasible Cinderella implementation that can correctly react to Stepmother's actions. The most apparent counterexample in this case is that the Stepmother is able to pour random amounts of water, and would be capable to overflow at least one bucket during every one of her turns.

\subsection{Formal Representation}
\label{sec:formals}
We use two disjoint sets, $state$ and $inputs$, to describe a system.
A straightforward and intuitive way to represent an \emph{implementation} is by
defining a \emph{transition system}, composed of an initial state
predicate $I(s)$ of type $state \to bool$, as well as a transition relation
$T(s,i,s')$ of type $state \to inputs \to state \to bool$.

Combining the above, we represent an Assume-Guarantee (AG) contract using a set
of \emph{assumptions}, $A: state \rightarrow inputs \rightarrow bool$,
and a set of \emph{guarantees} $G$. The latter is further decomposed into two
distinct subsets $G_I: state \rightarrow bool$ and $G_T: state \rightarrow
inputs \rightarrow state \rightarrow bool$. $G_I$ defines the set of valid
initial states, and $G_T$ contains constraints that need to be satisfied in
every transition between two states. An important note at this point is that we
we do not make any distinction between internal state variables and outputs in the
formalism. This alone allows us to use state variables to (in some cases)
simplify the specification of guarantees, since we do not expect a contract
to be always defined over all variables in the transition system.

Consequently, we can formally define a realizable contract, as one for which any
preceeding state $s$ can take a transition into a new state $s'$ that satisfies
the guarantees, assuming valid inputs. For a system to be ever-reactive, these
new states $s'$ should be further usable as preceeding states in a future
transition. States like $s$ and $s'$ are defined as being \textit{viable}, if
and only if:
\begin{align}
\begin{split}
  \viable(s) &= \\
  \forall s,i. \ & (A(s, i) \Rightarrow \exists s'.~ G_T(s, i,s')
\land \viable(s'))
\label{eq:viable}
\end{split}
\end{align}
This equation is recursive and we interpret it coinductively, i.e., as a
greatest fixed-point.
A necessary condition, finally, is that the set of viable states
intersects with the set of initial states. As such, to conclude that a contract
is realizable, we require that
\begin{equation}
\exists s. G_I(s) \land \viable(s)
\label{eq:nonempty}
\end{equation}

The intuition behind our proposed algorithm in this paper relies on the
discovery of a greatest fixpoint that only contains viable states. In the case where a fixpoint is computed, we proceed by extracting a witnessing collection of reactions that are, by construction, guaranteed to satisfy the specification. To achieve both the fixpoint generation, as well as the witness extraction, we depend on \aeval, the solver for $\forall\exists$-formulas.

\subsection{\textit{AE-VAL}}
\label{sec:aeval}

\begin{figure}[!t]
\centering
\includegraphics[width=3.5in]{aeval_invalid}
\caption{Region of validity computed for an example requiring \aeval to iterate two times.}
\label{fg:aeval}
\end{figure}

\andreas{I believe the section can be a bit longer. I also think that the keyphrase ``region of validity'' needs to stand out more. Finally, a proof for Lemma 1, or at least an outline of it would be greatly appreciated.}

\aeval~\cite{fedyukovich2015automated} is an algorithm to decide validity and extract Skolem functions.
It takes as input a formula of the form $\forall s \,.\,  A(s) \Rightarrow \exists s' . G(s,s')$, 
where $A(s)$ has only existential\andreas{you meant universal here, right?} quantifiers, and $G(s,s')$ is quantifier-free.
%
\andreas{I suppose that the partitions P are the ones named T in the Figure. I think we should keep the same name or the reader might get confused.}
While deciding the validity, \aeval iteratively enumerates models of $A(s) \land G (s, s')$ and groups them into a set of partitions $\{P_i(s)\}$, such that each $P_i(s) \Rightarrow \exists s' . G (s, s')$.
We say that after $n$ iterations, \aeval establishes a formula $R_n(s) \eqdef \bigvee_{i=1}^n P_i(s)$ which is by definition an under-approximation of $\exists s' . G (s, s')$.

If after $n$ iterations, it happens that $A(s) \Rightarrow R_n(s)$ then $\forall s \,.\,  A(s) \Rightarrow \exists s' . G(s,s')$ is valid, and \aeval generates a Skolem function as described in~\cite{katis2016synthesis}.
Alternatively, if $A(s) \land  G (s, s') \land \neg{R_n (s, s')}$ is unsatisfiable, then $A(s) \land \neg G (s, s')$ is satisfiable, or equivalently $\forall s \,.\,  A(s) \Rightarrow \exists s' . G(s,s')$ is invalid (see an example in Figure~\ref{fg:aeval}).
In both cases, we say that $A(s) \land R_n(s)$ is a \emph{region of validity}, meaning that $\forall s \,.\,  A(s) \land R_n(s) \Rightarrow \exists s' . G(s,s')$ is valid by construction.

\begin{lemma}
If formula $\forall s \,.\,  A(s) \Rightarrow \exists s' . G(s,s')$ is invalid, and $A(s) \land R_n(s)$ is the region of validity, then there is no other formula $S(s)$ such that $A(s) \land R_n(s) \Rightarrow S(s)$ and $\forall s \,.\,  S(s) \Rightarrow \exists s' . G(s,s')$.

\label{lem:subset}
\end{lemma}


\section{Validity-Guided Synthesis from Assume-Guarantee Contracts}
\label{sec:synthesis}


In this section we define Assume-Guarantee contracts (Sect.~\ref{sec:pre}),
describe the validity-guided approach we take towards synthesizing
implementations (Sect.~\ref{sec:synth}),
and finally illustrate how it works using the popular cinderella-stepmother
game (Sect.~\ref{sec:example}).
%Finally, we enrich our formal definitions with an informal proof of the
%algorithm's correctness in terms of the successfully synthesized
%implementations.

\subsection{Assume-Guarantee Contracts}
\label{sec:pre}

For the purposes of this paper, we focus our interest in a mainstream variation
for representing system requirements, using an \textit{Assume-Guarantee
Contract}. Requirements in this format contain two main types of constraints.
The \emph{assumptions} of the contract restrict the possible inputs that the
environment can provide to the system, while the \emph{guarantees} are used to
describe what is considered a safe reaction of the system to the outside world. 

A simple example is the contract with assumption $A = \{x\neq
y\}$ and guarantee $G = \{x \leq y \Rightarrow z =
\textit{true}, x \geq y \Rightarrow z = \textit{false}\}$. Variables
$x$ and $y$ are the designated inputs while $z$ is the output. This is a
well-defined contract, since by the assumption $A$, it is always the case that
$x \neq y$, therefore at least one implementation exists, which, for example
sets $z$ to true if $x < y$ and false otherwise. An alternative
valid implementation could set $z$ to false if $x > y$, and true otherwise. The
proof of existence of such an implementation is the main concept behind the
\emph{realizability} problem, while the automated construction of a witness
implementation is the main focus of \emph{program synthesis}.


It is apparent that the example contract above is therefore \emph{realizable},
and an efficient synthesis procedure would be capable of providing at least one
implementation. Nevertheless, it is important to consider a variation of the
example, where $A = \emptyset$. This is a practical case of an
\emph{unrealizable} contract, as there is no feasible implementation that can
correctly react to the environment assigning values to $x$ and $y$, such that
$x = y$.

\subsection{Formal Representation}
\label{sec:formals}
We use two disjoint sets, $state$ and $inputs$, to describe a system.
A straightforward and intuitive way to represent an \emph{implementation} is by
defining a \emph{transition system}, composed of an initial state
predicate $I(s)$ of type $state \to bool$, as well as a transition relation
$T(s,i,s')$ of type $state \to inputs \to state \to bool$.

Combining the above, we represent an Assume-Guarantee (AG) contract using a set
of \emph{assumptions}, $A: state \rightarrow inputs \rightarrow bool$,
and a set of \emph{guarantees} $G$. The latter is further decomposed into two
distinct subsets $G_I: state \rightarrow bool$ and $G_T: state \rightarrow
inputs \rightarrow state \rightarrow bool$. $G_I$ defines the set of valid
initial states, and $G_T$ contains constraints that need to be satisfied in
every transition between two states. An important note at this point is that we
we do not make any distinction between internal state variables and outputs in the
formalism. This alone allows us to use state variables to (in some cases)
simplify specification of guarantees, since we do not expect a contract
to be always defined over all variables in the transition system.

Consequently, we can formally define a realizable contract, as one for which any
preceeding state $s$ can make a transition into a new state $s'$ that satisfies
the guarantees, assuming valid inputs. For a system to be ever-reactive, these
new states $s'$ should be further usable as preceeding states in a future
transition. States like $s$ and $s'$ are defined as being \textit{viable}, if
and only if:

\begin{equation}
\forall s,i.~ A(s, i) \land \viable(s) \Rightarrow \exists s'.~ G_T(s, i,s')
\land \viable(s')
\label{eq:viable}
\end{equation}

A necessary condition, finally, is that the set of viable states has to
intersect with the set of initial states. As such, to conclude that a contract
is realizable, we require that

\begin{equation}
\viable(s) \land G_I(s) \neq \emptyset
\label{eq:nonempty}
\end{equation}

Therefore, the intuition behind our proposed algorithm relies on the discovery
of a greatest fixpoint that only contains viable states.


\subsection{Validity-Guided Synthesis from Contracts}
\label{sec:synth}

\begin{algorithm2e}[t]
\SetAlgoSkip{}
\SetKwFor{For}{for}{do}{}
\KwOut{$Result: \{\realizable, \unrealizable\}$, 
%\textcolor{red}{\init: state},
\skolems: Skolem Function for implementation or cex
}
\BlankLine
$\skolems \gets null$; \\
%$InitResult \gets $\sc{Sat?}$(G_I)$; \\
% \uIf(\label{alg:initState}){$(\isUnsat(InitResult))$}
% 	{%
% 		\Return
% 		\unrealizable, $\emptyset$, $\langle \rangle$;%
% 	}
%\textcolor{red}{$\init \gets InitResult.model$;} \\
$F(s) \gets true$;\\	
\While{true}{
$\phi \gets \forall s,i. F(s) \land A(s,i) \Rightarrow \exists s'. G_{T}(s,i,s')
\land F(s')$;
\\
$aevalResult \gets \aeval(\phi)$;\\
\uIf(\label{alg:returnSat}){$(\isValid(aevalResult))$}
{
	$\skolems \gets (aevalResult.\skolems)$;\\
	\uIf(){$G_{I}(s) \land F(s) \neq false$}
	{
	\Return \realizable, %\textcolor{red}{\init},
	 \skolems;
	}\uElse(){
	 \Return \unrealizable;
	}
}
\uElse
	{%
		$Q(s,i) \gets aevalResult.validSubset$;\\
		\uIf{$Q(s,i) = false$}{
			\Return \unrealizable;
		}
		$\phi' \gets \forall s. F(s) \Rightarrow \exists i. A(s,i) \land \lnot
		Q(s,i)$;\\
		$aevalResult' \gets \aeval(\phi')$;\\		
		\uIf(\label{alg:returnSat}){$(\isValid(aevalResult'))$}
		{
			$\skolems \gets (aevalResult'.\skolems)$;\\			
			\Return \unrealizable, \skolems;
		}\uElse{
			$W(s) \gets aevalResult'.validSubset$;\\
			$F(s) = F(s) \land \lnot W(s)$;
		}
}
}
\caption{Validity-Guided Synthesis}
\label{alg:synthesis}
\end{algorithm2e}

The main contribution presented in this paper, is a novel idea that effectively
uses the information provided by \textit{regions of validity} to compute a
greatest fixpoint of safe states. In our context, this fixpoint is not only
usable as a proof to the realizability of a specification, but also leads to the
construction of a witness that can be translated with a straightforward process
into a functional and efficient implementation.

Algorithm~\ref{alg:synthesis} shows this validity-guided technique.
We initialize the process by defining the fixpoint $F(s)$ to be equal to $true$.
Eventually, the algorithm attempts to converge to such a fixpoint $F(s)$ that
only contains viable states, considering Equation~\ref{eq:viable}. We therefore
construct the formula $\phi$, and provide it as an input to \aeval, an efficient
Skolemizer for $\forall\exists$ formulas. \aeval is particularly focused on
determining the validity of $\phi$. If the formula is valid, then a witness
\textit{Skolem} is constructed, containing valid assignments to the
existentially quantified variables of $\phi$. In the context of viability, this
witness is capable of providing viable states that can be used as a safe
reaction, considering the precedence of a viable state and an input that
satisfies the assumptions.

If $\phi$ is not true for every possible assignment of the universally
quantified variables, \aeval provides an exact subset of $F(s) \land A(s,i)$, namely
$Q(s,i)$, which, if plugged in the original left-hand side of $\phi$, makes the
resulting formula valid. We will refer to such subsets as \textit{regions of
validity}.

At this point, one could falsely assume that replacing $F(s) \land A(s,i)$ with
$Q(s,i)$ is sufficient to solve our problem, and use the resulting witness as a
candidate implementation. This is not the case however, as $Q(s,i)$ is a subset
of both state and input variables. As such, it may contain further constraints
over the contract's inputs. This would lead to implementations that only
consider a subset of the original assumptions of the contract, with no
pre-defined strategy for the rest of the originally valid inputs.
Fortunately, we can exploit \aeval's capability of providing regions of validity
towards eliminating this issue.

The main concept to properly refine $F(s)$, is to extract a region of validity
that only involves constraints over state variables. To achieve this, we ask for
the validity of the formula $\phi' \gets \forall s. F(s) \Rightarrow \exists
i. A(s,i) \land \lnot Q(s,i)$. If $\phi'$ is a valid formula, then for any
assignment of the state variables $s$, we have a valid input (i.e. that
satisfies the assumptions), for which the states are taken outside of the region
of validity $Q(s,i)$. This is a case of an unrealizable contract, as no state is
safe in this context. On the other hand, if $phi'$ is not valid, \aeval computes
a new region of validity, namely $W(s)$. The new region is a strict subset of
$F(s)$, is described using constraints over state variables only, and entails
the existence of unsafe states in $F(s)$, considering valid inputs.

Having this new region of validity, we can finally refine $F(s)$, by conjucting
to it the negation of $W(s)$. Despite the fact that we now have a refined
candidate fixpoint, we are not yet done, as $\lnot W(s)$ is not an exact region
of validity, with respect to $Q(s,i)$. As such there might still be states in
$Q(s,i)$ that are not covered by $\lnot W(s)$. Therefore, we reiterate the
process by repeating the top-level \aeval query, with $F(s) = F(s) \land \lnot
W(s)$. Eventually, we either reach a greatest fixpoint $F(s)$ that effectively
describes the set of viable states, or reach the case where $F(s) = false$, and
declare our contract to be unrealizable.

\aeval's effectiveness in providing witnesses to the
satisfiability of $\forall\exists$ formulas is also exploited in terms of the
tool providing concrete counterexamples to unrealizable contracts, using line 19
in Algorithm~\ref{alg:synthesis}. In this particular case, if $\phi'$ is a valid
formula, we can extract a witness that can be essentially used as a test case to
demonstrate the specification's unrealizability. The witness contains
certain assignments to input variables, for which the condition of viability does not
hold, for any state. We leave the specifics regarding the meaning and usability
of such counterexamples as potential future work.

From the previous, it is straightforward to show that
Algorithm~\ref{alg:synthesis} is complete, since it either terminates by
constructing a fixpoint, or when the computed region of validity is the empty
set \textit{false}. To prove its soundness regarding results, we require a
proof that the algorithm always computes a fixpoint, containing only state
variable assignments that lead to the satisfiability of $\forall\exists$
formulas that follow the form of Equation~\ref{eq:viable}. To achieve this, we
use the infamous fixed point theorem that was first stated by Alfred Tarski
in~\cite{tarski1955lattice}.

\begin{lemma} Consider the system
$\mathfrak{U} = \langle S, T \rangle$. With $S$, we denote the set of
subsets of the orignal state space, such that each subset contains assignments that lead to the
satisfiability of Equation~\ref{eq:viable}. By $T$, we refer to the
transition relation between any two states, that establishes a partial order
on $S$. Then $\mathfrak{U}$ is a complete lattice, where every subset $B \subseteq
S$ has a greatest lower bound  $\glb = \cap B$  and a least upper
bound $\lub = \cup B$.
\label{lem:lattice}
\end{lemma}
\begin{proof}
Considering the partial order that is established by T, it is straightforward
to show that all subsets $B$ of $S$ contain a \glb and a \lub. These
are respectively, the states which have no preceeding state in $B$ other than
possible ones in the \glb, and the states from which we take a transition into
a new state that's either in the \lub, or outside of $B$. For the special case
where $B = S$, we have that $\glb = false$ and $\lub = true$.
\end{proof}

\begin{lemma} Algorithm~\ref{alg:synthesis} is a monotonic function on $S$ to
$S$.
\label{lem:monotonicity}
\end{lemma}
\begin{proof}
The algorithm recursively reduces $S$, attempting to reach a fixed point
at which $S$ only contains state assignments that lead to the satisfiability of
Equation~\ref{eq:viable}. As such, it can be considered as an isotone function
$f$, where, for every pair $(B,A)$ with $B \subseteq A \subseteq S$, we have that
$f(B) \subseteq f(A)$.
\end{proof}

\begin{theorem}[Soundness of Fixpoint]
The set $P$ of all fixpoints in Algorithm~\ref{alg:synthesis} is non
empty, and the system $\langle P, T \rangle$ is a complete lattice.
\label{thm:fixpoint}
\end{theorem}
\begin{proof}
The proof relies on Theorem 1 in~\cite{tarski1955lattice}. Considering
Lemmas~\ref{lem:lattice} and~\ref{lem:monotonicity}, we satisfy the first two
conditions of Tarski's theorem. When the specification is realizable, a
fixpoint is reached by Algorithm~\ref{alg:synthesis}, since each consecutive
attempt to further refine $F(s)$ results in the same set. On the other hand, if
the specification is unrealizable, the algorithm returns the fixpoint $F(s) = false$. Therefore, the
set $P$ of all fixpoints in Algorithm~\ref{alg:synthesis} contains at least two
fixpoints.

Since all three conditions of Tarski's Fixed Point theorem are satisfied by our
solution, we can conclude that $P$ is a non empty set, while the system
$\langle P, T \rangle$ is a complete lattice, as it contains a \lub, which is
the solution to a realizable contract, while $\glb = false$, and corresponds to
the solution for an unrealizable contract.
\end{proof}


\subsection{An Illustrative Example}
\label{sec:example}
\section{Implementation}
\label{sec:impl}

The implementation of the algorithm has been added as an
unofficial feature to \jkind~\cite{jkind}, a Java implementation of the
\textsc{KIND} model checker. \jkind already unofficially supports synthesis,
using a k-inductive approach, named \jsyn. For clarity, we named
our validity-guided technique \jsynvg. Specification is described using the
Lustre language~\cite{lustrev6}, which functions as an intermediate representation to the Architecture Analysis and Design Language (\textsc{AADL}).
The latter is a high-level specification and analysis language with which
contracts are expressed, using the Assume-Guarantee Reasoning (\textsc{AGREE})
framework~\cite{NFM2012:CoGaMiWhLaLu}.
%\footnote{An unofficial release of \jkind
%including our synthesis algorithm is available to download at
%https://github.com/andrewkatis/jkind-1/tree/synthesis. \aeval needs to be
%installed separately from https://github.com/grigoryfedyukovich/aeval.}.
During the internal process, \jsynvg translates Lustre specifications to
the SMT-LIB language, with which the $\forall\exists$-formulas, regions of
validity, as well as the witnesses are expressed. The underlying solver that is
used is the \aeval Skolemizer, which currently supports $\forall\exists$-formulas expressed using real (LRA) and integer (LIA) arithmetic, as well as
their combination (LIRA).
%
For realizable specification, and considering Algorithm~\ref{alg:synthesis}, we
recursively use regions of validity to block out unsafe states from the search
space.
Eventually, we reach a fixpoint that is sufficient to use in order to extract
a Skolem function that witnesses the realizability of the specification. Given
the generated function, what remains is its translation
into an efficient and practical implementation. To achieve this we use
\smtlibtoc, a tool that has been specifically developed to translate
similar \aeval Skolem functions to C implementations. The choice of C for
the target language is not due to limitations, but mostly to provide a complete
comparison against the previous work on \jsyn.
The Skolem functions are simple enough to be used as an intermediate
representation to any mainstream programming language.

%In future work,
%we intend to further improve on this, and
%many other aspects of the compiler's performance as part of an individual
%future work.



%%% Local Variables:
%%% mode: latex
%%% TeX-master: "document"
%%% End:

\section{Experimental Results}
\label{sec:results}

\begin{figure}[!t]
\centering
\subfloat[Performance Comparison]{
\includegraphics[width=3in]{overhead-crop}
\label{fg:performance}}
%\hspace{+6.5em}

\subfloat[Size of Synthesized
Implementations]{\includegraphics[width=3in]{loc-crop}
\label{fg:size}}

\subfloat[Performance of Synthesized Implementations]{\includegraphics[width=3in]{performance-crop}
\label{fg:implperformance}}
\caption{Experimental results}
\label{fg:results}
\end{figure}

In this section we evaluate \jsynvg by synthesizing implementations
for 110 contracts that originate from a broad variety of contexts.~\footnote{An
anonymized repository containing the benchmark contracts can be found at
\url{https://tinyurl.com/mjtyxae }. We will replace this repository with the official for the camera-ready
version of the paper.} Almost half of them, 54, come from a collection of contracts that were initally used for the verification of existing handwritten implementations. The second biggest collection in our suite contains 52 contracts that correspond to various development projects, such as a Quad-Redundant Flight Control System, a Generic Patient Controlled Analgesia infusion pump, as well as a collection of contracts
for a Microwave model, written by graduate students as part of a software
engineering class. The remaining nine models contain variations of the
Cinderella-Stepmother game and the example in Section~\ref{sec:pre}.

Since \jkind already supports synthesis through \jsyn, we were able to directly
compare \jsynvg against \jsyn's k-inductive approach.
The comparison spans over multiple factors, including
algorithm performance, number of problems solved and, finally, performance
of C implementations through the translation of witnesses using \smtlibtoc. We
ran the experiments using a computer with Intel Core i3-4010U 1.70GHz CPU and
16GB RAM.

\begin{table}[!t]
\centering
\caption{Benchmark Statistics}
\label{tbl:stats}
\begin{tabular}{@{}lll@{}}
\toprule
 & \jsyn & \jsynvg \\ \midrule
Problems solved & 110 & \textbf{115} \\
Performance (avg - seconds) & 2.71 & \textbf{1.24} \\
Performance (max - seconds) & 159.05 & \textbf{64.89} \\
Implementation Size (avg - Lines of Code) & 74.45 & \textbf{69.77} \\
Implementation Size (max - Lines of Code) & 2382 & \textbf{2062} \\
Implementation Performance (avg - ms) & \textbf{56.57} & 58.47 \\
Implementation Performance (max - ms) & \textbf{464.359} & 485.197 \\
\bottomrule
\end{tabular}
\end{table}


\begin{table*}[!t]
\centering
\caption{Cinderella-Stepmother results}
\label{tbl:cindtbl}
\begin{tabular}{|c|c|c|c|c|c|}
\hline
 & \multicolumn{3}{c|}{\jsynvg} & \multicolumn{2}{c|}{\textsc{ConSynth}~\cite{beyene2014constraint}} \\ \hline
 & Implementation Size (LoC) & Implementation Performance (ms) & Time & Time (Z3) & Time (Barcelogic) \\ \hline
Cinderella (C = 3) & 92 & 262.84 & 35s & \multirow{3.2s} & \multirow{1.2s} \\ \cline{1-4}
Cinderella2 (C = 3) & 222 & 309.24 & 2m9s &  &  \\ \hline
Cinderella (C = 2) & 102 & 196.57 & 24s & \multirow{1m52s} & \multirow{1m52s} \\ \cline{1-4}
Cinderella2 (C = 2) & 272 & 230.16 & 2m9s &  &  \\ \hline
\end{tabular}
\end{table*}

A listing of the statistics that we tracked while running experiments is
presented in Table~\ref{tbl:stats}.
Figure~\ref{fg:performance} shows the time allocated by \jsyn and \jsynvg to solve each problem, with \jsynvg
outperforming \jsyn across the board, often times by a margin greater than
50\%. Figure~\ref{fg:size} on the other hand, depicts small differences in the
overall size between the synthesized implementations. While it would be
reasonable to conclude that there are no noticable improvements, the bigger
picture is different. The key factor that is not apparent from this figure, is the length of the k-inductive proof. For the majority of the benchmarks in the suite, \jsyn proves their realizability by constructing proofs of length $k=0$, which essentially means
that the entire space of states is an inductive invariant. As such, \jsynvg
also manages to synthesize implementations without requiring any refinement
process. For these cases, both algorithms generate a single Skolem function. In the general case though, the size of \jsyn solutions is directly
dependent on $k$, since each implementation is composed of $k$ Skolem
functions ($k-1$ to initialize the system, and one last for the inductive step),
where the equivalent solution from \jsynvg is always just one.
Figure~\ref{fg:size} hints towards this intuition, through a handful of spikes
in \jsyn implementation size. Despite this, we also noticed cases where \jsyn
implementations are shorter. This provides us with another interesting
observation regarding the formulation of the problem for $k=0$ proofs. In
these cases, \jsyn proves the existence of viable states, starting from a set
of \textit{pre-initial} states, where the contract does not need to hold. This
has direct implications to the way that the $\forall\exists$-formulas are
constructed in \jsyn's underlying machinery, where the assumptions are ``baked''
into the transition relation, affecting thus the creation of Model-Based
Projections by \aeval.

 One last statistic that we tracked was the performance of the synthesized C
 implementations, which can be seen in Figure~\ref{fg:implperformance}. For this purpose, we translated the
 generated witnesses from \jsyn and \jsynvg solutions using
 \smtlibtoc under the same set of options. Table~\ref{tbl:stats} shows that
 while \jsyn implementations are faster, the difference is minuscule on average. 
This may be in part because \jsyn creates separate skolem functions for the initial evaluation (when \%init is true) and subsequent evaluations, whereas currently \jsynvg uses a single function for both the initial and subsequent steps.  We are considering specialization of the generated \jsynvg functions based on \%init, as well as several other optimizations of the generated code in future work.

 
% The deciding factor in this context is the
% difference in complexity of the Model-Based Projections that get generated by
% \aeval using \jsyn and \jsynvg, with the latter versions containing richer
% expressions, mainly due to the refinement process.

Figure~\ref{fg:results} does not cover the entirety of the
benchmark suite. From the original 110 problems, five of them cannot be
solved by \jsyn's k-inductive approach. Four of these files are variations of
the cinderella-stepmother game that we described in Section~\ref{sec:example}, using different representations of the game, as well as two different values
for the bucket capacity (2 and 3). Using the variation that we included in the
aforementioned section as an input to \jsyn, we receive an ``unrealizable'' answer, with the counterexample shown
in Figure~\ref{fg:cex}. Reading through the feedback provided by \jsyn, it is
apparent that the underlying SMT solver is incapable of choosing the correct
buckets to empty, leading eventually to a state where an overflow occurs for the
third bucket. As we already discussed though, a winning strategy exists for the
cinderella, as long as the bucket capacity $c$ is between 1.5 and 3. This
provides an excellent demonstration regarding the inherent weakness of \jsyn
in providing sound ``unrealizable'' results. \jsynvg's validity-guided approach,
on the other hand, was able to prove the realizability for these contracts, as
well as synthesize an implementation for each.

\begin{figure}[!t]
\centering
 \begin{Verbatim}[fontsize=\scriptsize]
 ++++++++++++++++++++++++++++++++++++++++++++++++++++++++++
      UNREALIZABLE || K = 6 || Time = 2.017s
                 Step
      variable      0    1      2      3      4      5
      INPUTS
      i1            0    0      0 0.416* 0.944* 0.666*
      i2            1    0 0.083* 0.083*      0 0.055*
      i3            0    1 0.305*    0.5 0.027* 0.194*
      i4            0    0 0.611*      0      0 0.027*
      i5            0    0      0      0 0.027* 0.055*

      OUTPUTS
      e             1    3      1      5      4      5

      NODE OUTPUTS
      guarantee   true true   true   true   true  false

      NODE LOCALS
      b1            0    0      0 0.416* 1.361* 0.666*
      b2            0    0 0.083* 0.166* 0.166* 0.222*
      b3            0    1 1.305* 1.805* 1.833* 2.027*
      b4            0    0 0.611* 0.611*      0 0.027*
      b5            0    0      0      0 0.027* 0.055*

      * display value has been truncated
 +++++++++++++++++++++++++++++++++++++++++++++++++++++++++
 \end{Verbatim}
\caption{Spurious counterexample for Cinderella-Stepmother example using \jsyn}

\label{fg:cex}
\end{figure}

Table~\ref{tbl:cindtbl} shows how \jsynvg performed against the four contracts describing the Cinderella-Stepmother game. We used two different interpretations for the game, and exercised both for the cases where the bucket capacity $C$ is equal to 2 and 3. The performance is heavily affected when we exercise the variation for $C=2$, but \jsynvg is still able to synthesize a winning region for Cinderella. Regarding the synthesized implementations, their size remains analogous to the difficulty of the problem, in conjunction with the complexity of the program (Cinderella2 contains more local variables and a helper function to empty buckets). Despite this, the implementation performance remains at the same levels across all implementations. Finally for reference, the table contains the results from the template-based approach followed in \textsc{Consynth}~\cite{beyene2014constraint}. From the results, it is apparent that providing templates dramatically increases the performance of the underlying synthesis procedure. Nevertheless, the automated approach in \jsynvg is able to synthesize solutions for the same problems in a reasonable time margin, when compared to \textsc{ConSynth}.


Overall, \jsynvg's validity-guided approach provides significant advantages
over the k-inductive technique followed in \jsyn, and effectively expands
\jkind's solving capabilities regarding specification realizability. On top of that, it provides an efficient ``hands-off'' approach that is capable of solving complex games.
The most significant contribution, however, is the applicability of this approach, as it is not tied to a specific environment since it can be extended to support more
theories, as well as categories of specification.

\section{Related Work}
\label{sec:related}

\iffalse
The alternative name to program synthesis is Church's problem, since the description of the problem was first described by Church in 1963~\cite{church1962logic}. Research in this field of program synthesis attributes began in the 1970s, when Manna and Waldinger~\cite{manna1971toward} first introduced a synthesis procedure using principles of theorem proving. Almost two decades later, Pnueliand Rosner~\cite{pnueli1989synthesis} first formally described the
implementability of reactive systems, considering first order logic formulas
that stem from temporal specifications. In the same work, they provided a
complete approach to synthesize finite-state implementations through the
construction of deterministic Rabin automata.

In the recent years, program synthesis has enjoyed a vast variety of
contributions under numerous contexts. Gulwani~\cite{gulwani2010dimensions}
presented an extended survey, hinting future research directions. Synthesis
algorithms have been proposed for simple LTL specification~\cite{Bohy12,Tini03}
subsets of it~\cite{Klein10,ehlers2010symbolic,cheng2016structural}, as well as under other temporal logics~\cite{monmege2016real,Hamza10}, such as SIS~\cite{Aziz95}.
Chatterjee and Henzinger~\cite{Chatterjee07} proposed a novel component-based approach using the notion of Assume-Guarantee contracts.
\fi
The work presented in this paper is closely related to approaches that attempt
to construct infinite-state implementations. Some focus on the continuous
interaction of the user with the underlying machinery, either through the use of
templates~\cite{srivastava2013template,beyene2014constraint}, or environments where the user attempts to guide the solver by choosing reactions from a collection of different
interpretations~\cite{ryzhyk2016developing}.
In contrast, our approach is completely automatic and does not require
human ingenuity to find a solution.
Most importantly, the user
does not need to be deeply familiar with the problem at hand.
\iffalse
 A ``hands-off'' approach has been proposed in the past by Katis \textit{et
al.}~\cite{gacek2015towards,katis2016towards,katis2016synthesis}. The work is
based on the concept of extracting collections of reactions that witness the satisfaction of a $k$-inductive proof on the contract's realizability.
In Section~\ref{sec:results} we were able to effectively show in practice how a
validity-guided approach significantly improves upon this work, by using a
fixpoint-generating technique rather than the principle of $k$-induction.
\fi

Iterative strengthening of candidate formulas is also used in abductive inference~\cite{dillig2013inductive}
 %Dillig \textit{et al.} showed how abduction can be used
of loop invariants.
Their approach generates candidate invariants as maximum universal subsets (MUS) of quantifier-free formulas of the form $\phi \Rightarrow \psi$.
%   via a refinement process that generates candidate invariants in the form of maximum universal subsets (MUS).
While a MUS may be sufficient to prove validity, it may also mislead the invariant search, so the authors use a backtracking procedure that discovers new subsets while avoiding spurious results. By comparison, in our approach the regions of validity are maximal and therefore backtracking is not required.  More importantly, reactive synthesis requires mixed-quantifier formulas, and it requires that inputs are unconstrained (other than by the contract assumptions), so substantial modifications to the MUS algorithm would be necessary to apply the approach of~\cite{dillig2013inductive} for reactive synthesis.  %In our work, the ability to generate precise regions of validity lead to a very simple refinement process that is solely based on the input variables being existentially quantified.

The concept of synthesizing implementations by discovering fixpoints was mostly
inspired by the IC3 / PDR~\cite{bradley2011sat,een2011efficient}, which was first introduced
in the context of verification. Work from Cimatti \textit{et al.} effectively
applied this idea for the parameter synthesis in the
\textsc{HyComp} model checker~\cite{DBLP:conf/fmcad/CimattiGMT13, cimatti2015hycomp}.
Discovering fixpoints to synthesize reactive designs was first
extensively covered by Piterman \textit{et al.}~\cite{piterman2006synthesis}
who proved that the problem can be solved in cubic time for the class of GR(1) specifications.
The algorithm requires the discovery of least fixpoints for the state variables,
each one covering a greatest fixpoint of the input variables. If the specification
is realizable, the entirety of the input space is covered by the greatest fixpoints.
In contrast, our approach computes a single greatest fixpoint over the system's outputs and avoids the partitioning of the input space.  As the tools use different notations and support different logical fragments, practical comparisons are not straightforward, and thus are left for the future.

More recently, Preiner \textit{et al}. presented work on model synthesis~\cite{preiner2017counterexample}, %that constructs candidate witnesses constructed and their validity is checked with respect to the specification. Internally,
that employs a counterexample-guided refinement process~\cite{reynolds2015counterexample}
to construct and check candidate models.
Internally, it relies on enumerative learning, a syntax-based technique that enumerates expressions, checks their validity against
ground test cases, and proceeds to generalize the expressions by constructing larger ones.
In contrast, our approach is syntax-insensitive in terms of generating regions of validity.
\iffalse
On top of that, CEGMS has only been tested against the theory of Boolean Vectors (BV, and BV$_{LNIRA}$), while \jsynvg supports the mixed theories of linear integer and real arithmetic.
\fi
In general, enumeration techniques such as the one used in \textsc{ConSynth}'s underlying E-HSF engine~\cite{beyene2014constraint} is not an optimal strategy for our class of problems, since the witnesses constructed for the most complex contracts are described by nested if-then-else expressions of depth (i.e. number of branches) 10-20, a point at which space explosion is difficult to handle since the number of candidate solutions is large.

\section{Conclusion and Future Work}
\label{sec:conclusion}

In this paper, we contributed an approach to program synthesis guided by the
proofs of realizability of Assume-Guarantee contracts.
To check realizability, it performs k-induction-based reasoning to decide validity of a set of $\forall\exists$-formulas.
Whenever a contract is proven realizable, it further employs the Skolemization procedure and extracts a fine-grained witness to the realizability.
Those Skolem functions are finally encoded into a desirable implementation.
We implemented the technique in the \jsynvg tool and evaluated it for the set
of Lustre models of different complexity.
The experimental results provided fruitful conclusions on the overall efficacy of the the approach.

To the best of our knowledge our work is the first complete attempt on providing
a synthesis algorithm based on the principle of k-induction using infinite
theories. The ability to express contracts that support ideas from many
categories of specifications, such as template-based and temporal properties,
increases the potential applicability of this work to multiple subareas on
synthesis research.

In future work, we plan to exploit the connections of our approach with
%eventually switch to a more efficient algorithm, where we endorse the core
Property Directed Reachability~\cite{komuravelli2014smt} more closely. %~\cite{een2011efficient,bradley11}.
%using efficient ways to further enhance the algorithm's performance through the
%use of implicit abstractions~\cite{cimatti2014ic3} to further reduce the search
%space of the algorithm.
%This will also help our original work on realizability checking, by improving the unsoundness of our unrealizable results. 
Another promising idea here is the use of Inductive Validity Cores (IVCs)~\cite{Ghass16}, whose main purpose is to effectively pinpoint the absolutely necessary model elements in a generated proof. We can potentially use
the information provided by IVCs as a preprocessing tool to reduce the size of
the original specification, and hopefully the complexity of the realizability
proof. Finally, various optimizations can be implemented on top of both
\aeval and \textsc{SMTLib2C} to produce smaller implementations.
\section{Data Availability Statement}
\label{sec:artifact}

The datasets generated during and/or analysed during the current study are available in the figshare repository: https://doi.org/10.6084/m9.figshare.5904904~\cite{katis2018tacasartifact}


%\subsubsection*{\textbf{Acknowledgments}}
%This work was funded by DARPA and AFRL under contract FA8750-12-9-0179 (Secure
% Mathematically-Assured Composition of Control Models), and by NASA under contract NNA13AA21C (Compositional Verification of Flight Critical Systems), and by NSF under grant CNS-1035715 (Assuring the safety, security, and reliability of medical device cyber physical systems).
% \IEEEtriggeratref{25}
\bibliography{document}
\bibliographystyle{splncs03}
\end{document}
